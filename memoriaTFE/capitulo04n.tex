\documentclass[12pt,a4paper,]{book}
\def\ifdoblecara{} %% set to true
\def\ifprincipal{} %% set to true
\let\ifprincipal\undefined %% set to false
\def\ifcitapandoc{} %% set to true
\let\ifcitapandoc\undefined %% set to false
\usepackage{lmodern}
% sin fontmathfamily
\usepackage{amssymb,amsmath}
\usepackage{ifxetex,ifluatex}
%\usepackage{fixltx2e} % provides \textsubscript %PLLC
\ifnum 0\ifxetex 1\fi\ifluatex 1\fi=0 % if pdftex
  \usepackage[T1]{fontenc}
  \usepackage[utf8]{inputenc}
\else % if luatex or xelatex
  \ifxetex
    \usepackage{mathspec}
  \else
    \usepackage{fontspec}
  \fi
  \defaultfontfeatures{Ligatures=TeX,Scale=MatchLowercase}
\fi
% use upquote if available, for straight quotes in verbatim environments
\IfFileExists{upquote.sty}{\usepackage{upquote}}{}
% use microtype if available
\IfFileExists{microtype.sty}{%
\usepackage{microtype}
\UseMicrotypeSet[protrusion]{basicmath} % disable protrusion for tt fonts
}{}
\usepackage[margin = 2.5cm]{geometry}
\usepackage{hyperref}
\hypersetup{unicode=true,
            pdfauthor={Nombre Completo Autor},
              pdfborder={0 0 0},
              breaklinks=true}
\urlstyle{same}  % don't use monospace font for urls
%
\usepackage[usenames,dvipsnames]{xcolor}  %new PLLC
\IfFileExists{parskip.sty}{%
\usepackage{parskip}
}{% else
\setlength{\parindent}{0pt}
\setlength{\parskip}{6pt plus 2pt minus 1pt}
}
\setlength{\emergencystretch}{3em}  % prevent overfull lines
\providecommand{\tightlist}{%
  \setlength{\itemsep}{0pt}\setlength{\parskip}{0pt}}
\setcounter{secnumdepth}{5}
% Redefines (sub)paragraphs to behave more like sections
\ifx\paragraph\undefined\else
\let\oldparagraph\paragraph
\renewcommand{\paragraph}[1]{\oldparagraph{#1}\mbox{}}
\fi
\ifx\subparagraph\undefined\else
\let\oldsubparagraph\subparagraph
\renewcommand{\subparagraph}[1]{\oldsubparagraph{#1}\mbox{}}
\fi

%%% Use protect on footnotes to avoid problems with footnotes in titles
\let\rmarkdownfootnote\footnote%
\def\footnote{\protect\rmarkdownfootnote}


  \title{}
    \author{Nombre Completo Autor}
      \date{18/11/2021}


%%%%%%% inicio: latex_preambulo.tex PLLC


%% UTILIZA CODIFICACIÓN UTF-8
%% MODIFICARLO CONVENIENTEMENTE PARA USARLO CON OTRAS CODIFICACIONES


%\usepackage[spanish,es-nodecimaldot,es-noshorthands]{babel}
\usepackage[spanish,es-nodecimaldot,es-noshorthands,es-tabla]{babel}
% Ver: es-tabla (en: https://osl.ugr.es/CTAN/macros/latex/contrib/babel-contrib/spanish/spanish.pdf)
% es-tabla (en: https://tex.stackexchange.com/questions/80443/change-the-word-table-in-table-captions)
\usepackage[spanish, plain, datebegin,sortcompress,nocomment,
noabstract]{flexbib}
 
\usepackage{float}
\usepackage{placeins}
\usepackage{fancyhdr}
% Solucion: ! LaTeX Error: Command \counterwithout already defined.
% https://tex.stackexchange.com/questions/425600/latex-error-command-counterwithout-already-defined
\let\counterwithout\relax
\let\counterwithin\relax
\usepackage{chngcntr}
%\usepackage{microtype}  %antes en template PLLC
\usepackage[utf8]{inputenc}
\usepackage[T1]{fontenc} % Usa codificación 8-bit que tiene 256 glyphs

%\usepackage[dvipsnames]{xcolor}
%\usepackage[usenames,dvipsnames]{xcolor}  %new
\usepackage{pdfpages}
%\usepackage{natbib}




% Para portada: latex_paginatitulo_mod_ST02.tex (inicio)
\usepackage{tikz}
\usepackage{epigraph}
\input{portadas/latex_paginatitulo_mod_ST02_add.sty}
% Para portada: latex_paginatitulo_mod_ST02.tex (fin)

% Para portada: latex_paginatitulo_mod_OV01.tex (inicio)
\usepackage{cpimod}
% Para portada: latex_paginatitulo_mod_OV01.tex (fin)

% Para portada: latex_paginatitulo_mod_OV03.tex (inicio)
\usepackage{KTHEEtitlepage}
% Para portada: latex_paginatitulo_mod_OV03.tex (fin)

\renewcommand{\contentsname}{Índice}
\renewcommand{\listfigurename}{Índice de figuras}
\renewcommand{\listtablename}{Índice de tablas}
\newcommand{\bcols}{}
\newcommand{\ecols}{}
\newcommand{\bcol}[1]{\begin{minipage}{#1\linewidth}}
\newcommand{\ecol}{\end{minipage}}
\newcommand{\balertblock}[1]{\begin{alertblock}{#1}}
\newcommand{\ealertblock}{\end{alertblock}}
\newcommand{\bitemize}{\begin{itemize}}
\newcommand{\eitemize}{\end{itemize}}
\newcommand{\benumerate}{\begin{enumerate}}
\newcommand{\eenumerate}{\end{enumerate}}
\newcommand{\saltopagina}{\newpage}
\newcommand{\bcenter}{\begin{center}}
\newcommand{\ecenter}{\end{center}}
\newcommand{\beproof}{\begin{proof}} %new
\newcommand{\eeproof}{\end{proof}} %new
%De: https://texblog.org/2007/11/07/headerfooter-in-latex-with-fancyhdr/
% \fancyhead
% E: Even page
% O: Odd page
% L: Left field
% C: Center field
% R: Right field
% H: Header
% F: Footer
%\fancyhead[CO,CE]{Resultados}

%OPCION 1
% \fancyhead[LE,RO]{\slshape \rightmark}
% \fancyhead[LO,RE]{\slshape \leftmark}
% \fancyfoot[C]{\thepage}
% \renewcommand{\headrulewidth}{0.4pt}
% \renewcommand{\footrulewidth}{0pt}

%OPCION 2
% \fancyhead[LE,RO]{\slshape \rightmark}
% \fancyfoot[LO,RE]{\slshape \leftmark}
% \fancyfoot[LE,RO]{\thepage}
% \renewcommand{\headrulewidth}{0.4pt}
% \renewcommand{\footrulewidth}{0.4pt}
%%%%%%%%%%
\usepackage{calc,amsfonts}
% Elimina la cabecera de páginas impares vacías al finalizar los capítulos
\usepackage{emptypage}
\makeatletter

%\definecolor{ocre}{RGB}{25,25,243} % Define el color azul (naranja) usado para resaltar algunas salidas
\definecolor{ocre}{RGB}{0,0,0} % Define el color a negro (aparece en los teoremas

%\usepackage{calc} 


%era if(csl-refs) con dolares
% metodobib: true


\usepackage{lipsum}

%\usepackage{tikz} % Requerido para dibujar formas personalizadas

%\usepackage{amsmath,amsthm,amssymb,amsfonts}
\usepackage{amsthm}


% Boxed/framed environments
\newtheoremstyle{ocrenumbox}% % Theorem style name
{0pt}% Space above
{0pt}% Space below
{\normalfont}% % Body font
{}% Indent amount
{\small\bf\sffamily\color{ocre}}% % Theorem head font
{\;}% Punctuation after theorem head
{0.25em}% Space after theorem head
{\small\sffamily\color{ocre}\thmname{#1}\nobreakspace\thmnumber{\@ifnotempty{#1}{}\@upn{#2}}% Theorem text (e.g. Theorem 2.1)
\thmnote{\nobreakspace\the\thm@notefont\sffamily\bfseries\color{black}---\nobreakspace#3.}} % Optional theorem note
\renewcommand{\qedsymbol}{$\blacksquare$}% Optional qed square

\newtheoremstyle{blacknumex}% Theorem style name
{5pt}% Space above
{5pt}% Space below
{\normalfont}% Body font
{} % Indent amount
{\small\bf\sffamily}% Theorem head font
{\;}% Punctuation after theorem head
{0.25em}% Space after theorem head
{\small\sffamily{\tiny\ensuremath{\blacksquare}}\nobreakspace\thmname{#1}\nobreakspace\thmnumber{\@ifnotempty{#1}{}\@upn{#2}}% Theorem text (e.g. Theorem 2.1)
\thmnote{\nobreakspace\the\thm@notefont\sffamily\bfseries---\nobreakspace#3.}}% Optional theorem note

\newtheoremstyle{blacknumbox} % Theorem style name
{0pt}% Space above
{0pt}% Space below
{\normalfont}% Body font
{}% Indent amount
{\small\bf\sffamily}% Theorem head font
{\;}% Punctuation after theorem head
{0.25em}% Space after theorem head
{\small\sffamily\thmname{#1}\nobreakspace\thmnumber{\@ifnotempty{#1}{}\@upn{#2}}% Theorem text (e.g. Theorem 2.1)
\thmnote{\nobreakspace\the\thm@notefont\sffamily\bfseries---\nobreakspace#3.}}% Optional theorem note

% Non-boxed/non-framed environments
\newtheoremstyle{ocrenum}% % Theorem style name
{5pt}% Space above
{5pt}% Space below
{\normalfont}% % Body font
{}% Indent amount
{\small\bf\sffamily\color{ocre}}% % Theorem head font
{\;}% Punctuation after theorem head
{0.25em}% Space after theorem head
{\small\sffamily\color{ocre}\thmname{#1}\nobreakspace\thmnumber{\@ifnotempty{#1}{}\@upn{#2}}% Theorem text (e.g. Theorem 2.1)
\thmnote{\nobreakspace\the\thm@notefont\sffamily\bfseries\color{black}---\nobreakspace#3.}} % Optional theorem note
\renewcommand{\qedsymbol}{$\blacksquare$}% Optional qed square
\makeatother



% Define el estilo texto theorem para cada tipo definido anteriormente
\newcounter{dummy} 
\numberwithin{dummy}{section}
\theoremstyle{ocrenumbox}
\newtheorem{theoremeT}[dummy]{Teorema}  % (Pedro: Theorem)
\newtheorem{problem}{Problema}[chapter]  % (Pedro: Problem)
\newtheorem{exerciseT}{Ejercicio}[chapter] % (Pedro: Exercise)
\theoremstyle{blacknumex}
\newtheorem{exampleT}{Ejemplo}[chapter] % (Pedro: Example)
\theoremstyle{blacknumbox}
\newtheorem{vocabulary}{Vocabulario}[chapter]  % (Pedro: Vocabulary)
\newtheorem{definitionT}{Definición}[section]  % (Pedro: Definition)
\newtheorem{corollaryT}[dummy]{Corolario}  % (Pedro: Corollary)
\theoremstyle{ocrenum}
\newtheorem{proposition}[dummy]{Proposición} % (Pedro: Proposition)


\usepackage[framemethod=default]{mdframed}



\newcommand{\intoo}[2]{\mathopen{]}#1\,;#2\mathclose{[}}
\newcommand{\ud}{\mathop{\mathrm{{}d}}\mathopen{}}
\newcommand{\intff}[2]{\mathopen{[}#1\,;#2\mathclose{]}}
\newtheorem{notation}{Notation}[chapter]


\mdfdefinestyle{exampledefault}{%
rightline=true,innerleftmargin=10,innerrightmargin=10,
frametitlerule=true,frametitlerulecolor=green,
frametitlebackgroundcolor=yellow,
frametitlerulewidth=2pt}


% Theorem box
\newmdenv[skipabove=7pt,
skipbelow=7pt,
backgroundcolor=black!5,
linecolor=ocre,
innerleftmargin=5pt,
innerrightmargin=5pt,
innertopmargin=10pt,%5pt
leftmargin=0cm,
rightmargin=0cm,
innerbottommargin=5pt]{tBox}

% Exercise box	  
\newmdenv[skipabove=7pt,
skipbelow=7pt,
rightline=false,
leftline=true,
topline=false,
bottomline=false,
backgroundcolor=ocre!10,
linecolor=ocre,
innerleftmargin=5pt,
innerrightmargin=5pt,
innertopmargin=10pt,%5pt
innerbottommargin=5pt,
leftmargin=0cm,
rightmargin=0cm,
linewidth=4pt]{eBox}	

% Definition box
\newmdenv[skipabove=7pt,
skipbelow=7pt,
rightline=false,
leftline=true,
topline=false,
bottomline=false,
linecolor=ocre,
innerleftmargin=5pt,
innerrightmargin=5pt,
innertopmargin=10pt,%0pt
leftmargin=0cm,
rightmargin=0cm,
linewidth=4pt,
innerbottommargin=0pt]{dBox}	

% Corollary box
\newmdenv[skipabove=7pt,
skipbelow=7pt,
rightline=false,
leftline=true,
topline=false,
bottomline=false,
linecolor=gray,
backgroundcolor=black!5,
innerleftmargin=5pt,
innerrightmargin=5pt,
innertopmargin=10pt,%5pt
leftmargin=0cm,
rightmargin=0cm,
linewidth=4pt,
innerbottommargin=5pt]{cBox}

% Crea un entorno para cada tipo de theorem y le asigna un estilo 
% con ayuda de las cajas coloreadas anteriores
\newenvironment{theorem}{\begin{tBox}\begin{theoremeT}}{\end{theoremeT}\end{tBox}}
\newenvironment{exercise}{\begin{eBox}\begin{exerciseT}}{\hfill{\color{ocre}\tiny\ensuremath{\blacksquare}}\end{exerciseT}\end{eBox}}				  
\newenvironment{definition}{\begin{dBox}\begin{definitionT}}{\end{definitionT}\end{dBox}}	
\newenvironment{example}{\begin{exampleT}}{\hfill{\tiny\ensuremath{\blacksquare}}\end{exampleT}}		
\newenvironment{corollary}{\begin{cBox}\begin{corollaryT}}{\end{corollaryT}\end{cBox}}	

%	ENVIRONMENT remark
\newenvironment{remark}{\par\vspace{10pt}\small 
% Espacio blanco vertical sobre la nota y tamaño de fuente menor
\begin{list}{}{
\leftmargin=35pt % Indentación sobre la izquierda
\rightmargin=25pt}\item\ignorespaces % Indentación sobre la derecha
\makebox[-2.5pt]{\begin{tikzpicture}[overlay]
\node[draw=ocre!60,line width=1pt,circle,fill=ocre!25,font=\sffamily\bfseries,inner sep=2pt,outer sep=0pt] at (-15pt,0pt){\textcolor{ocre}{N}}; \end{tikzpicture}} % R naranja en un círculo (Pedro)
\advance\baselineskip -1pt}{\end{list}\vskip5pt} 
% Espaciado de línea más estrecho y espacio en blanco después del comentario


\newenvironment{solutionExe}{\par\vspace{10pt}\small 
\begin{list}{}{
\leftmargin=35pt 
\rightmargin=25pt}\item\ignorespaces 
\makebox[-2.5pt]{\begin{tikzpicture}[overlay]
\node[draw=ocre!60,line width=1pt,circle,fill=ocre!25,font=\sffamily\bfseries,inner sep=2pt,outer sep=0pt] at (-15pt,0pt){\textcolor{ocre}{S}}; \end{tikzpicture}} 
\advance\baselineskip -1pt}{\end{list}\vskip5pt} 

\newenvironment{solutionExa}{\par\vspace{10pt}\small 
\begin{list}{}{
\leftmargin=35pt 
\rightmargin=25pt}\item\ignorespaces 
\makebox[-2.5pt]{\begin{tikzpicture}[overlay]
\node[draw=ocre!60,line width=1pt,circle,fill=ocre!55,font=\sffamily\bfseries,inner sep=2pt,outer sep=0pt] at (-15pt,0pt){\textcolor{ocre}{S}}; \end{tikzpicture}} 
\advance\baselineskip -1pt}{\end{list}\vskip5pt} 

\usepackage{tcolorbox}

\usetikzlibrary{trees}

\theoremstyle{ocrenum}
\newtheorem{solutionT}[dummy]{Solución}  % (Pedro: Corollary)
\newenvironment{solution}{\begin{cBox}\begin{solutionT}}{\end{solutionT}\end{cBox}}	


\newcommand{\tcolorboxsolucion}[2]{%
\begin{tcolorbox}[colback=green!5!white,colframe=green!75!black,title=#1] 
 #2
 %\tcblower  % pone una línea discontinua
\end{tcolorbox}
}% final definición comando

\newtcbox{\mybox}[1][green]{on line,
arc=0pt,outer arc=0pt,colback=#1!10!white,colframe=#1!50!black, boxsep=0pt,left=1pt,right=1pt,top=2pt,bottom=2pt, boxrule=0pt,bottomrule=1pt,toprule=1pt}



\mdfdefinestyle{exampledefault}{%
rightline=true,innerleftmargin=10,innerrightmargin=10,
frametitlerule=true,frametitlerulecolor=green,
frametitlebackgroundcolor=yellow,
frametitlerulewidth=2pt}





\newcommand{\betheorem}{\begin{theorem}}
\newcommand{\eetheorem}{\end{theorem}}
\newcommand{\bedefinition}{\begin{definition}}
\newcommand{\eedefinition}{\end{definition}}

\newcommand{\beremark}{\begin{remark}}
\newcommand{\eeremark}{\end{remark}}
\newcommand{\beexercise}{\begin{exercise}}
\newcommand{\eeexercise}{\end{exercise}}
\newcommand{\beexample}{\begin{example}}
\newcommand{\eeexample}{\end{example}}
\newcommand{\becorollary}{\begin{corollary}}
\newcommand{\eecorollary}{\end{corollary}}


\newcommand{\besolutionExe}{\begin{solutionExe}}
\newcommand{\eesolutionExe}{\end{solutionExe}}
\newcommand{\besolutionExa}{\begin{solutionExa}}
\newcommand{\eesolutionExa}{\end{solutionExa}}


%%%%%%%%


% Caja Salida Markdown
\newmdenv[skipabove=7pt,
skipbelow=7pt,
rightline=false,
leftline=true,
topline=false,
bottomline=false,
backgroundcolor=GreenYellow!10,
linecolor=GreenYellow!80,
innerleftmargin=5pt,
innerrightmargin=5pt,
innertopmargin=10pt,%5pt
innerbottommargin=5pt,
leftmargin=0cm,
rightmargin=0cm,
linewidth=4pt]{mBox}	

%% RMarkdown
\newenvironment{markdownsal}{\begin{mBox}}{\end{mBox}}	

\newcommand{\bmarkdownsal}{\begin{markdownsal}}
\newcommand{\emarkdownsal}{\end{markdownsal}}


\usepackage{array}
\usepackage{multirow}
\usepackage{wrapfig}
\usepackage{colortbl}
\usepackage{pdflscape}
\usepackage{tabu}
\usepackage{threeparttable}
\usepackage{subfig} %new
%\usepackage{booktabs,dcolumn,rotating,thumbpdf,longtable}
\usepackage{dcolumn,rotating}  %new
\usepackage[graphicx]{realboxes} %new de: https://stackoverflow.com/questions/51633434/prevent-pagebreak-in-kableextra-landscape-table

%define el interlineado vertical
%\renewcommand{\baselinestretch}{1.5}

%define etiqueta para las Tablas o Cuadros
%\renewcommand\spanishtablename{Tabla}

%%\bibliographystyle{plain} %new no necesario


%%%%%%%%%%%% PARA USO CON biblatex
% \DefineBibliographyStrings{english}{%
%   backrefpage = {ver pag.\adddot},%
%   backrefpages = {ver pags.\adddot}%
% }

% \DefineBibliographyStrings{spanish}{%
%   backrefpage = {ver pag.\adddot},%
%   backrefpages = {ver pags.\adddot}%
% }
% 
% \DeclareFieldFormat{pagerefformat}{\mkbibparens{{\color{red}\mkbibemph{#1}}}}
% \renewbibmacro*{pageref}{%
%   \iflistundef{pageref}
%     {}
%     {\printtext[pagerefformat]{%
%        \ifnumgreater{\value{pageref}}{1}
%          {\bibstring{backrefpages}\ppspace}
%          {\bibstring{backrefpage}\ppspace}%
%        \printlist[pageref][-\value{listtotal}]{pageref}}}}
% 
%%% de kableExtra
\usepackage{booktabs}
\usepackage{longtable}
%\usepackage{array}
%\usepackage{multirow}
%\usepackage{wrapfig}
%\usepackage{float}
%\usepackage{colortbl}
%\usepackage{pdflscape}
%\usepackage{tabu}
%\usepackage{threeparttable}
\usepackage{threeparttablex}
\usepackage[normalem]{ulem}
\usepackage{makecell}
%\usepackage{xcolor}

%%%%%%% fin: latex_preambulo.tex PLLC


\begin{document}

\bibliographystyle{flexbib}



\raggedbottom

\ifdefined\ifprincipal
\else
\setlength{\parindent}{1em}
\pagestyle{fancy}
\setcounter{tocdepth}{4}
\tableofcontents

\fi

\ifdefined\ifdoblecara
\fancyhead{}{}
\fancyhead[LE,RO]{\scriptsize\rightmark}
\fancyfoot[LO,RE]{\scriptsize\slshape \leftmark}
\fancyfoot[C]{}
\fancyfoot[LE,RO]{\footnotesize\thepage}
\else
\fancyhead{}{}
\fancyhead[RO]{\scriptsize\rightmark}
\fancyfoot[LO]{\scriptsize\slshape \leftmark}
\fancyfoot[C]{}
\fancyfoot[RO]{\footnotesize\thepage}
\fi

\renewcommand{\headrulewidth}{0.4pt}
\renewcommand{\footrulewidth}{0.4pt}

\hypertarget{anuxe1lisis-exploratorio-de-datos}{%
\chapter{Análisis exploratorio de
datos}\label{anuxe1lisis-exploratorio-de-datos}}

En este capítulo se aplicarán distintos métodos numéricos y gráficos de
análisis de datos a la muestra generada siguiendo el procedimiento
detallado en el capítulo anterior. Se usarán principalmente técnicas de
estadística descriptiva para comprender las características del conjunto
de datos y extraer información útil para el problema que se intenta
abordar, predecir incendios forestales. Es importante tener presente que
se trata de datos correlados espacial y temporalmente, lo que hace
necesario el uso de métodos específicos para este tipo de datos. Los
objetivos de esta etapa son:

\begin{enumerate}
\def\labelenumi{\arabic{enumi}.}
\item
  Generar conocimiento sobre el conjunto de datos que nos permita
  evaluar la calidad de este, sin olvidar las limitaciones que ya se han
  comentado en la sección anterior.
\item
  Conocer, al menos de forma descriptiva, el impacto de cada variable en
  la variable objetivo. Este conocimiento será necesario para evaluar e
  interpretar los modelos que se construirán en la próxima sección.
\item
  Analizar las características de las distintas variables, de cara a
  usar posteriormente técnicas de preprocesamiento adecuadas para cada
  modelo.
\end{enumerate}

Antes de abordar el estudio detallado de cada una de las variables y las
relaciones entre estas, en la Figura \ref{fig:skim_datos} se recoge un
resumen de todo el conjunto de datos, sin incluir la columna de
geometría. En este resumen se puede observar que en el conjunto de datos
hay 4 tipos de variables (además de la variable \emph{geometry} que es
de tipo \emph{simple feature column POINT}, abreviado como
\emph{sfc\_POINT}): cadenas de caracteres, fechas, factores y variables
numéricas.

Se puede observar que hay registros en 749 municipios diferentes (de los
785 municipios de que hay en Andalucía). Probablemente el hecho de que
en algunos municipios no haya habido observaciones sea debido a los
datos faltantes. Las variables \emph{municipio} y \emph{cod\_municipio}
no se incorporarán a los modelos. De la misma forma, se puede ver que
hay observaciones en 3691 días diferentes.

El conjunto cuenta con 5 variables de tipo factor: \emph{fire} (la
variable objetivo), \emph{WD10M}, \emph{orientacion}, \emph{enp} y
\emph{uso\_suelo}; y con 18 variables numéricas. Aunque cada una de
ellas se analizará a continuación con detalle, ya cabe hacer algunos
comentarios:

\begin{itemize}
\tightlist
\item
  El \(38\%\) de las observaciones se encuentran en espacios de
  vegetación arbustiva y/o herbácea (código 32).
\item
  Como era de esperar, por la forma en la que se ha tomado la muestra,
  el conjunto está balanceado.
\item
  El \(81\%\) de las observaciones se encuentran fuera de Espacios
  Naturales Protegidos.
\item
  Todas las variables, salvo \emph{T2M} y \emph{curvatura}, son
  positivas y la mayoría de ellas presentan una marcada distribución
  asimétrica hacia la derecha.
\item
  Las variables muestran escalas muy diversas entre ellas, siendo
  \emph{GWETTOP} la que presenta menor desviación típica (\(0.145\)) y
  \emph{poblacion} la que tiene una desviación típica mayor (\(64453\)).
  Se evidencia la necesidad de incluir algún método de normalización de
  las variables en el preprocesamiento de los datos.
\end{itemize}

\begin{figure}[htb]
\centering
\includegraphics[width = \textwidth]{graficos/skim_datos.png}
\caption[Resumen numérico del conjunto de datos depurados]{Resumen numérico del conjunto de datos depurados. \textit{ Fuente: Elaboración propia empleando la función} \texttt{skim} \it del paquete "skimr" \citep{skimrpackage}.}
\label{fig:skim_datos}
\end{figure}

\hypertarget{distribuciuxf3n-de-la-variable-objetivo}{%
\section{Distribución de la variable
objetivo}\label{distribuciuxf3n-de-la-variable-objetivo}}

En primer lugar, se estudiará la distribución de la variable \emph{fire}
espacial y temporalmente.

\begin{figure}[htb]
\centering
\includegraphics[width = 0.75\textwidth]{graficos/distribucion_temporal_fire.png}
\caption[Distribución temporal de la variable objetivo]{Distribución temporal de la variable objetivo. \it Fuente: Elaboración propia.}
\label{fig:dist_temp_fire}
\end{figure}

En la Figura \ref{fig:dist_temp_fire} se muestran los histogramas de la
variable objetivo en función del día de la semana, del mes y del año,
respectivamente. En primero el de ellos se observa que, mientras que la
distribución de los casos negativos es uniforme entre los días de la
semana, en los casos positivos se aprecia un ligero aumento en el fin de
semana, especialmente en el sábado. Esto podría ser algo meramente
casual o deberse al hecho de que más personas van al campo durante el
fin de semana por motivos de ocio y se producen más desplazamientos, lo
que podría podría aumentar el riesgo de que se produzcan negligencias o
accidentes que desencadenen incendios forestales (en 2022, el 39.23\% de
las actuaciones forestales registradas fueron debidas a neglicencias u
accidentes \citep{INFOCA2022}). En el segundo histograma, se observa
como las observaciones se concentran en los meses de verano y en cada
mes hay una cantidad balanceada de muestras de ambas clases (esto es
fruto del proceso de muestreo de las observaciones negativas, que como
se ha explicado en la sección anterior, se ha llevado acabo asegurando
que la proporción de casos negativos en cada mes sea igual a la de los
casos positivos). En el tercer histograma es remarcable que, mientra las
observaciones negativas están uniformemente distribuidas entre los 20
años del estudio, las positivas muestran una disminución importante en
los años 2008 y 2010. En el año 2007 no hay observaciones positivas,
debido a que los 4 polígonos de incendios mayores de 100 \(ha\) que
había registrados ese año no disponían de la fecha de inicio del
incendio, por lo que no pudieron usarse para el estudio. Se desconoce la
causa del reducido número de incendios (mayores de 100 \(ha\)) en 2007,
2008 y 2010.

Dada la clara influencia del mes y la aparente influencia del día de la
semana en la aparición de incendios (al menos, con certeza, de aquellos
de la dimensión correspondiente al estudio), estas variables serán
incluidas en los modelos a través del procesamiento de la variable
\emph{date}. Al hacer esto se está hipotetizando que el efecto del mes o
del día de la semana va más allá de las características meteorológicas
propias de cada periodo, ya que también contiene información sobre otras
dimensiones, como las diferentes tendencias sociales durante el periodo
vacacional o los cambios en los movimientos de población durante el fin
de semana o las distintas estaciones. Dada la imposibilidad de medir
todas estas variables individualmente, se espera que al menos parte del
efecto que puedan tener sobre la aparición de incendios forestales quede
recogido a través de su estacionalidad.

\begin{figure}[h]
\centering
\includegraphics[width = 0.5\textwidth]{graficos/distribucion_espacial_fire.png}
\caption[Distribución espacial de la variable objetivo]{Distribución espacial de la variable objetivo. \it Fuente: Elaboración propia.}
\label{fig:dist_spat_fire}
\end{figure}

En la Figura \ref{fig:dist_spat_fire} se observa claramente cómo las
\(10.752\) muestras negativas están uniformemente distribuidas dentro de
los límites de la Comunidad Autónoma de Andalucía, mientras que las
\(10794\) muestras positivas se concentran a ambos lados de la cuenca
del río Guadalquivir, con una mayor densidad de observaciones en la
provincia de Huelva y en algunas zonas de la costa mediterránea (como ya
se apreciaba en la Figura \ref{fig:area_fuego}).

\hypertarget{anuxe1lisis-univariantes-de-las-variables-numuxe9ricas}{%
\section{Análisis univariantes de las variables
numéricas}\label{anuxe1lisis-univariantes-de-las-variables-numuxe9ricas}}

El análisis univariante de las variables numéricas se lleva a cabo desde
3 enfoques complementarios:

\begin{enumerate}
\def\labelenumi{\arabic{enumi}.}
\item
  A través de la los resúmenes numéricos recogidos en la Figura
  \ref{fig:skim_datos} y del análisis gráfico de los diagramas de caja y
  bigotes (Figura \ref{fig:boxplots}).
\item
  Estudiando la media mensual de cada variable en función de la variable
  \emph{fire}.
\item
  Analizando la distribución espacial de cada variable separando por mes
  si corresponde. Los gráficos correspondientes a este análisis se
  recogen en el {[}Apéndice A.1{]}{[}Gráficos espaciales EDA{]}.
\end{enumerate}

\begin{figure}[]
\centering
\includegraphics[width =\textwidth]{graficos/boxplots.png}
\caption[Diagrama de caja y bigotes de cada variable numérica en función de la variable objetivo]{Boxplot de cada variable numérica en función de la variable objetivo. \it Fuente: Elaboración propia.}
\label{fig:boxplots}
\end{figure}

En los \emph{boxplots} de las variables numéricas en función de la
variable \emph{fire} (Figura \ref{fig:boxplots}) destacan varios
aspectos. Por un lado, como ya se había comentado anteriormente, las
variables presentan escalas muy diferentes y la mayoría tiene una
marcada asimetría hacia la derecha. Por otro lado, es evidente la gran
cantidad de valores \emph{outliers} que se observan en los datos, lo que
tendrá implicaciones en los modelos que se construyan con ellos. Sin
embargo, es importante destacar que no se trata de observaciones
erróneas, sino que son inherentes a la naturaleza de los datos. Por
ejemplo, en el caso de la variable \emph{PRECTOTCORR} el valor máximo
observado es 46.06 \(mm\) en un día, un valor elevado que sin duda es
atípico en esta región de clima seco, pero sin embargo, posible. Es
también remarcable que todas las variables presentan una variabilidad
similar en ambos niveles del factor \emph{fire}, lo que indica que no
será un problema de clasificación trivial. \emph{A priori}, solo con los
diagramas de caja y bigotes y los resúmenes numéricos es difícil llegar
a más conclusiones, sin embargo, sí pueden observarse sutiles
diferencias entre las distribuciones de algunas variables para ambos
niveles del factor \emph{fire}.

Dada la naturaleza temporal de algunas variables, el análisis gráfico de
los \emph{boxplots} resulta insufiente. Con el fin de considerar la
componente estacional de las variables climáticas y de vegetación, se
estudiará, a continuación, la media mensual de cada una de estas
variables en función de la variable objetivo.

\begin{figure}[H]
\centering
\includegraphics[width = 0.5\textwidth]{graficos/T2M_mes.png}
\caption[Media mensual de $T2M$ en función de $fire$]{Media mensual de $T2M$ en función de $fire$. \it Fuente: Elaboración propia.}
\label{fig:T2M_mes}
\end{figure}

En la Figura \ref{fig:T2M_mes} se puede observar cómo en casi todos los
meses, la temperatura media mensual es superior en las observaciones en
las que se ha registrado un incendio forestal.

\begin{figure}[H]
\centering
\includegraphics[width =0.5\textwidth]{graficos/RH2M_mes.png}
\caption[Media mensual de $RH2M$ en función de  $fire$]{Media mensual de $RH2M$ en función de  $fire$. \it Fuente: Elaboración propia.}
\label{fig:RH2M_mes}
\end{figure}

En la Figura \ref{fig:RH2M_mes} se puede observar que en todos los meses
la media mensual de la humedad relativa del aire a 2 \(m\) sobre la
superficie es menor en las observaciones en las que se ha registrado un
incendio forestal. Sin embargo, las diferencias se reducen durante los
meses de verano, en los que la humedad presenta valores bajos en ambas
clases.

\begin{figure}[H]
\centering
\includegraphics[width =0.5\textwidth]{graficos/PRECTOTCORR_mes.png}
\caption[Media mensual de $PRECTOTCORR$ en función de $fire$]{Media mensual de $PRECTOTCORR$ en función de $fire$. \it Fuente: Elaboración propia.}
\label{fig:PRECTOTCORR_mes}
\end{figure}

En la Figura \ref{fig:PRECTOTCORR_mes} se observa una clara diferencia
en la media mensual de las precipitaciones diarias en función de si se
ha registrado o no un incendio forestal en la observación, siendo
significativamente mayor en este último caso.

\begin{figure}[H]
\centering
\includegraphics[width =0.5\textwidth]{graficos/GWETTOP_mes.png}
\caption[Media mensual de $GWETTOP$ en función de $fire$]{Media mensual de $GWETTOP$ en función de $fire$. \it Fuente: Elaboración propia.}
\label{fig:GWETTOP_mes}
\end{figure}

En la Figura \ref{fig:GWETTOP_mes}, que muestra la media mensual de la
humedad del suelo en función de si en esa observación se ha registrado o
no incendio, se observa un gráfico similar al de la humedad relativa del
aire, con valores medios más elevados en las observaciones en las que no
se han registrado incendio forestal. Sin embargo, también parece que las
diferencias son más reducidas durante la estación estival.

\begin{figure}[H]
\centering
\includegraphics[width = 0.5\textwidth]{graficos/WS10M_mes.png}
\caption[Media mensual de $WS10M$ en función de $fire$]{Media mensual de $WS10M$ en función de $fire$. \it Fuente: Elaboración propia.}
\label{fig:WS10M_mes}
\end{figure}

En la Figura \ref{fig:WS10M_mes} se observa cómo, durante todos los
meses, la media mensual de la velocidad del viento a 10 metros sobre la
superficie es mayor en los registros en los que ha habido un incendio
forestal. Esto es, de hecho, algo remarcable. Ya que los incendios que
se están considerando son incendios que han llegado a calcinar un área
superior a 100 \(ha\), es de esperar que se trate de incendios que se
dan bajo unas condiciones meteorológicas concretas que facilitan la
rápida propagación del fuego, dificultado su extinción temprana. Y dado
que se está considerando la fecha de inicio de los incendios, esto es lo
que podría estar viéndose en la Figura \ref{fig:WS10M_mes}.

\begin{figure}[H]
\centering
\includegraphics[width = 0.5\textwidth]{graficos/NDVI_mes.png}
\caption[Media mensual de $NDVI$ en función de $fire$]{Media mensual de $NDVI$ en función de $fire$. \it Fuente: Elaboración propia.}
\label{fig:NDVI_mes}
\end{figure}

Como se observa en la Figura \ref{fig:NDVI_mes}, las diferencias entre
los casos en los que se ha registrado incendio y los que no, en términos
del \emph{NDVI}, no están claras. Se recuerda que esta variable se
utiliza para cuantificar la cantidad y verdor de la vegetación, por lo
que es coherente que se observen valores medios más bajos en los meses
de verano.

En el {[}Apéndice A.1{]}{[}Gráficos espaciales EDA{]} se recogen los
gráficos espaciales y espacio-temporales de todas las variables
numéricas. En ellos se refleja cómo los valores de las variables en
estudio son coherentes con lo que cabría esperar de la realidad. Además,
permiten una comprensión mayor de la distribución espacial (y temporal)
de las variables en el área de estudio, lo que será útil de cara a
interpretar los modelos que se construyan.

\hypertarget{anuxe1lisis-multivariantes-de-las-variables-numuxe9ricas}{%
\section{Análisis multivariantes de las variables
numéricas}\label{anuxe1lisis-multivariantes-de-las-variables-numuxe9ricas}}

En la Figura \ref{fig:corrplot} se muestra un gráfico con las
correlaciones entre las variables. La interpretación es sencilla, cuanto
más intenso sea el color y cuanto mayor sea la excentricidad de la
elipse, mayor será la correlación (en valor absoluto) para ese par de
variables. El color de la elipse indica el signo del coeficiente de
correlación. De esta forma, se observa que las variables más
correlacionadas en la muestra son: \emph{T2M} con \emph{RH2M}
(negativamente, -0.71), \emph{T2M} con \emph{GWETTOP} (negativamente,
-0.69), \emph{GWETTOP} con \emph{R2HM} (positivamente, 0.68) y
\emph{poblacion} con \emph{dens\_poblacion} (positivamente, 0.63). Esto
es razonable, ya que cabe esperar que al aumentar la temperatura del
aire disminuya la humedad del aire y del suelo; que al aumentar la
humedad del suelo aumente también la del aire y viceversa; y que
municipios muy densamente poblados también tengan un elevado número de
habitantes. Cabe destacar que no se trata de valores alarmantes como
para considerar \emph{a priori} que las variables proporcionan
información redundante, susceptible de ser eliminadas en un paso de
ingeniería de características. Sin embargo, sí se probará a reducir la
dimensionalidad de los datos capturando la mayor parte de la varianza a
través de PCA.

\begin{figure}[h]
\centering
\includegraphics[width =0.7\textwidth]{graficos/corrplot.png}
\caption[Correlaciones entre variables numéricas]{Correlaciones entre variables numéricas. \it Fuente: Elaboración propia.}
\label{fig:corrplot}
\end{figure}

En la Figura \ref{fig:parcoord} se muestra el gráfico de coordenadas
paralelas de las variables tipificadas a una normal estándar, es decir,
restándoles la media y dividiendo por la desviación típica. Este gráfico
complementa la información de los \emph{boxplots}, pues refleja también
las relaciones entre las variables. Si bien es cierto que al tener un
número bastante elevado de observaciones el gráfico no es tan claro,
pueden hacerse algunas observaciones importantes.

En primer lugar, se observa que la variable con mayor variabilidad (una
vez tipificada) es PRECTOTCORR, que presenta bastantes valores atípicos,
todos ellos en observaciones en las que no se ha registrado incendio.
También destacan en este sentido \emph{dens\_poblacion} y
\emph{poblacion}, entre las que además puede observarse que no hay una
relación lineal clara (hay municipios con un elevado número de
habitantes pero con una densidad de población reducida y viceversa).
Además, puede verse que todas las variables tienen una marcada asimetría
positiva (salvo \emph{curvatura}, \emph{T2M} y \emph{NDVI}). Este
gráfico es útil también pues permite ver a qué clase de \emph{fire}
corresponden los valores más atípicos de cada variable. Por ejemplo: la
mayor parte de los valores más elevados de \emph{WS10M},
\emph{dist\_poblacion}, \emph{curvatura} y \emph{dist\_camino} se dan en
observaciones positivas, mientras que en \emph{PRECTOTCORR},
\emph{elevacion}, \emph{GWTTOP}, \emph{dist\_Carretera},
\emph{dist\_electr} y \emph{dist\_rios} sucede lo contrario.

\begin{figure}[h]
\centering
\includegraphics[width =\textwidth]{graficos/parcoord.png}
\caption[Gráfico de coordenadas paralelas de las variables numéricas tipificadas]{Gráfico de coordenadas paralelas de las variables numéricas tipificadas. \it Fuente: Elaboración propia.}
\label{fig:parcoord}
\end{figure}

Los resultados de aplicar análisis de componentes principales sobre la
matriz de correlaciones de las 18 variables numéricas se muestran en la
Figura \ref{fig:PCA}. Como se puede observar, se necesitan al menos 11
componentes principales para lograr explicar el 80\% de la varianza de
la muestra, y 14 para alcanzar el 90\% de la varianza de los datos.
Estos resultados se aplicarán más adelante en los modelos, pero a nivel
meramente explicativo ya indican que se trata de un conjunto de datos
complejo en cuanto a la dimensión real de estos.

\begin{figure}[h]
\centering
\includegraphics[width =\textwidth]{graficos/pca.png}
\caption[PCA sobre la matriz de correlaciones de las variables numéricas]{PCA sobre la matriz de correlaciones de las variables numéricas. \it Fuente: Elaboración propia.}
\label{fig:PCA}
\end{figure}

\hypertarget{anuxe1lisis-de-las-variables-categuxf3ricas}{%
\section{Análisis de las variables
categóricas}\label{anuxe1lisis-de-las-variables-categuxf3ricas}}

Las variables categóricas se analizarán a través de los histogramas de
cada variable en función de la variable \emph{fire} (Figura
\ref{fig:histogramas}).

En la variable \emph{WD10M} cabe destacar la escasez de observaciones
con dirección del viento norte. En el histograma no se observa una clara
relación de esta variable con la variable objetivo, aunque entre las
observaciones con viento con dirección sur o suroeste hay más
observaciones negativas y entre las que tienen dirección noroeste o este
hay una mayor presencia de observaciones positivas.

En el caso de la variable \emph{orientación}, la relación tampoco está
clara, aunque puede verse una mayor proporción de observaciones
positivas en las superficies con orientación sur (sureste, sur y
suroeste).

En términos de la variable \emph{enp} por si sola no se observan
diferencias significativas entre ambas clases.

La variable \emph{uso\_suelo} sí que muestra una distribución
marcadamente diferenciada entre ambas clases. La mayoría de las
observaciones positivas se dan en espacios de vegetación arbustiva y/o
herbácea (código 32), clase en la que hay casi el doble de observaciones
positivas que negativas. En tierras de labor (código 21) y cultivos
permanentes (código 22) la proporción de observaciones negativas es
mucho mayor, mientras que en zonas agrícolas heterogéneas (código 24) y
en espacios abiertos con poca o sin vegetación (código 33) hay una mayor
presencia de observaciones positivas dentro de la muestra. También es
relevante el hecho de que casi la totalidad de las observaciones se
encuentran en zonas agrícolas y en zonas forestales (que se corresponden
con los códigos comenzados por 2 o 3, respectivamente), mientras que en
las demás clases la proporción de observaciones es mucho menor
(\(3.5\%\) de total). Es por ello que antes de construir los modelos,
todas las categorías de uso de suelo que no se corresponden con zonas
agrícolas o forestales (es decir, todas cuyo código no comienza por 2 o
3) se agruparán en el nivel \emph{Otro}. En la Figura
\ref{fig:uso_suelo_spat} del {[}Apéndice A1{]}{[}Gráficos espaciales
EDA{]} puede observarse la distribución espacial de esta variable.

\begin{figure}[]
\centering
\includegraphics[width =\textwidth]{graficos/histogramas.png}
\caption[Histogramas de las variables categóricas en función de $fire$]{Histogramas de las variables categóricas en función de $fire$. \it Fuente: Elaboración propia.}
\label{fig:histogramas}
\end{figure}

Prueba {[}Gráficos espaciales EDA{]}{[}{]}

\bibliography{bib/library.bib,bib/paquetes.bib}


%


\end{document}
