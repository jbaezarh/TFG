\documentclass[12pt,a4paper,]{book}
\def\ifdoblecara{} %% set to true
\def\ifprincipal{} %% set to true
\let\ifprincipal\undefined %% set to false
\def\ifcitapandoc{} %% set to true
\let\ifcitapandoc\undefined %% set to false
\usepackage{lmodern}
% sin fontmathfamily
\usepackage{amssymb,amsmath}
\usepackage{ifxetex,ifluatex}
%\usepackage{fixltx2e} % provides \textsubscript %PLLC
\ifnum 0\ifxetex 1\fi\ifluatex 1\fi=0 % if pdftex
  \usepackage[T1]{fontenc}
  \usepackage[utf8]{inputenc}
\else % if luatex or xelatex
  \ifxetex
    \usepackage{mathspec}
  \else
    \usepackage{fontspec}
  \fi
  \defaultfontfeatures{Ligatures=TeX,Scale=MatchLowercase}
\fi
% use upquote if available, for straight quotes in verbatim environments
\IfFileExists{upquote.sty}{\usepackage{upquote}}{}
% use microtype if available
\IfFileExists{microtype.sty}{%
\usepackage{microtype}
\UseMicrotypeSet[protrusion]{basicmath} % disable protrusion for tt fonts
}{}
\usepackage[margin = 2.5cm]{geometry}
\usepackage{hyperref}
\hypersetup{unicode=true,
            pdfauthor={Nombre Completo Autor},
              pdfborder={0 0 0},
              breaklinks=true}
\urlstyle{same}  % don't use monospace font for urls
%
\usepackage[usenames,dvipsnames]{xcolor}  %new PLLC
\IfFileExists{parskip.sty}{%
\usepackage{parskip}
}{% else
\setlength{\parindent}{0pt}
\setlength{\parskip}{6pt plus 2pt minus 1pt}
}
\setlength{\emergencystretch}{3em}  % prevent overfull lines
\providecommand{\tightlist}{%
  \setlength{\itemsep}{0pt}\setlength{\parskip}{0pt}}
\setcounter{secnumdepth}{5}
% Redefines (sub)paragraphs to behave more like sections
\ifx\paragraph\undefined\else
\let\oldparagraph\paragraph
\renewcommand{\paragraph}[1]{\oldparagraph{#1}\mbox{}}
\fi
\ifx\subparagraph\undefined\else
\let\oldsubparagraph\subparagraph
\renewcommand{\subparagraph}[1]{\oldsubparagraph{#1}\mbox{}}
\fi

%%% Use protect on footnotes to avoid problems with footnotes in titles
\let\rmarkdownfootnote\footnote%
\def\footnote{\protect\rmarkdownfootnote}


  \title{}
    \author{Nombre Completo Autor}
      \date{18/11/2021}


%%%%%%% inicio: latex_preambulo.tex PLLC


%% UTILIZA CODIFICACIÓN UTF-8
%% MODIFICARLO CONVENIENTEMENTE PARA USARLO CON OTRAS CODIFICACIONES


%\usepackage[spanish,es-nodecimaldot,es-noshorthands]{babel}
\usepackage[spanish,es-nodecimaldot,es-noshorthands,es-tabla]{babel}
% Ver: es-tabla (en: https://osl.ugr.es/CTAN/macros/latex/contrib/babel-contrib/spanish/spanish.pdf)
% es-tabla (en: https://tex.stackexchange.com/questions/80443/change-the-word-table-in-table-captions)
\usepackage[spanish, plain, datebegin,sortcompress,nocomment,
noabstract]{flexbib}
 
\usepackage{float}
\usepackage{placeins}
\usepackage{fancyhdr}
% Solucion: ! LaTeX Error: Command \counterwithout already defined.
% https://tex.stackexchange.com/questions/425600/latex-error-command-counterwithout-already-defined
\let\counterwithout\relax
\let\counterwithin\relax
\usepackage{chngcntr}
%\usepackage{microtype}  %antes en template PLLC
\usepackage[utf8]{inputenc}
\usepackage[T1]{fontenc} % Usa codificación 8-bit que tiene 256 glyphs

%\usepackage[dvipsnames]{xcolor}
%\usepackage[usenames,dvipsnames]{xcolor}  %new
\usepackage{pdfpages}
%\usepackage{natbib}




% Para portada: latex_paginatitulo_mod_ST02.tex (inicio)
\usepackage{tikz}
\usepackage{epigraph}
\renewcommand\epigraphflush{flushright}
\renewcommand\epigraphsize{\normalsize}
\setlength\epigraphwidth{0.7\textwidth}

\definecolor{titlepagecolor}{cmyk}{1,.60,0,.40}

%\DeclareFixedFont{\titlefont}{T1}{ppl}{b}{it}{0.5in}

% \makeatletter
% \def\printauthor{%
%     {\large \@author}}
% \makeatother
% \author{%
%     Author 1 name \\
%     Department name \\
%     \texttt{email1@example.com}\vspace{20pt} \\
%     Author 2 name \\
%     Department name \\
%     \texttt{email2@example.com}
%     }

% The following code is borrowed from: https://tex.stackexchange.com/a/86310/10898

\newcommand\titlepagedecoration{%
\begin{tikzpicture}[remember picture,overlay,shorten >= -10pt]

\coordinate (aux1) at ([yshift=-15pt]current page.north east);
\coordinate (aux2) at ([yshift=-410pt]current page.north east);
\coordinate (aux3) at ([xshift=-4.5cm]current page.north east);
\coordinate (aux4) at ([yshift=-150pt]current page.north east);

\begin{scope}[titlepagecolor!40,line width=12pt,rounded corners=12pt]
\draw
  (aux1) -- coordinate (a)
  ++(225:5) --
  ++(-45:5.1) coordinate (b);
\draw[shorten <= -10pt]
  (aux3) --
  (a) --
  (aux1);
\draw[opacity=0.6,titlepagecolor,shorten <= -10pt]
  (b) --
  ++(225:2.2) --
  ++(-45:2.2);
\end{scope}
\draw[titlepagecolor,line width=8pt,rounded corners=8pt,shorten <= -10pt]
  (aux4) --
  ++(225:0.8) --
  ++(-45:0.8);
\begin{scope}[titlepagecolor!70,line width=6pt,rounded corners=8pt]
\draw[shorten <= -10pt]
  (aux2) --
  ++(225:3) coordinate[pos=0.45] (c) --
  ++(-45:3.1);
\draw
  (aux2) --
  (c) --
  ++(135:2.5) --
  ++(45:2.5) --
  ++(-45:2.5) coordinate[pos=0.3] (d);   
\draw 
  (d) -- +(45:1);
\end{scope}
\end{tikzpicture}%
}

% Para portada: latex_paginatitulo_mod_ST02.tex (fin)

% Para portada: latex_paginatitulo_mod_OV01.tex (inicio)
\usepackage{cpimod}
% Para portada: latex_paginatitulo_mod_OV01.tex (fin)

% Para portada: latex_paginatitulo_mod_OV03.tex (inicio)
\usepackage{KTHEEtitlepage}
% Para portada: latex_paginatitulo_mod_OV03.tex (fin)

\renewcommand{\contentsname}{Índice}
\renewcommand{\listfigurename}{Índice de figuras}
\renewcommand{\listtablename}{Índice de tablas}
\newcommand{\bcols}{}
\newcommand{\ecols}{}
\newcommand{\bcol}[1]{\begin{minipage}{#1\linewidth}}
\newcommand{\ecol}{\end{minipage}}
\newcommand{\balertblock}[1]{\begin{alertblock}{#1}}
\newcommand{\ealertblock}{\end{alertblock}}
\newcommand{\bitemize}{\begin{itemize}}
\newcommand{\eitemize}{\end{itemize}}
\newcommand{\benumerate}{\begin{enumerate}}
\newcommand{\eenumerate}{\end{enumerate}}
\newcommand{\saltopagina}{\newpage}
\newcommand{\bcenter}{\begin{center}}
\newcommand{\ecenter}{\end{center}}
\newcommand{\beproof}{\begin{proof}} %new
\newcommand{\eeproof}{\end{proof}} %new
%De: https://texblog.org/2007/11/07/headerfooter-in-latex-with-fancyhdr/
% \fancyhead
% E: Even page
% O: Odd page
% L: Left field
% C: Center field
% R: Right field
% H: Header
% F: Footer
%\fancyhead[CO,CE]{Resultados}

%OPCION 1
% \fancyhead[LE,RO]{\slshape \rightmark}
% \fancyhead[LO,RE]{\slshape \leftmark}
% \fancyfoot[C]{\thepage}
% \renewcommand{\headrulewidth}{0.4pt}
% \renewcommand{\footrulewidth}{0pt}

%OPCION 2
% \fancyhead[LE,RO]{\slshape \rightmark}
% \fancyfoot[LO,RE]{\slshape \leftmark}
% \fancyfoot[LE,RO]{\thepage}
% \renewcommand{\headrulewidth}{0.4pt}
% \renewcommand{\footrulewidth}{0.4pt}
%%%%%%%%%%
\usepackage{calc,amsfonts}
% Elimina la cabecera de páginas impares vacías al finalizar los capítulos
\usepackage{emptypage}
\makeatletter

%\definecolor{ocre}{RGB}{25,25,243} % Define el color azul (naranja) usado para resaltar algunas salidas
\definecolor{ocre}{RGB}{0,0,0} % Define el color a negro (aparece en los teoremas

%\usepackage{calc} 


%era if(csl-refs) con dolares
% metodobib: true


\usepackage{lipsum}

%\usepackage{tikz} % Requerido para dibujar formas personalizadas

%\usepackage{amsmath,amsthm,amssymb,amsfonts}
\usepackage{amsthm}


% Boxed/framed environments
\newtheoremstyle{ocrenumbox}% % Theorem style name
{0pt}% Space above
{0pt}% Space below
{\normalfont}% % Body font
{}% Indent amount
{\small\bf\sffamily\color{ocre}}% % Theorem head font
{\;}% Punctuation after theorem head
{0.25em}% Space after theorem head
{\small\sffamily\color{ocre}\thmname{#1}\nobreakspace\thmnumber{\@ifnotempty{#1}{}\@upn{#2}}% Theorem text (e.g. Theorem 2.1)
\thmnote{\nobreakspace\the\thm@notefont\sffamily\bfseries\color{black}---\nobreakspace#3.}} % Optional theorem note
\renewcommand{\qedsymbol}{$\blacksquare$}% Optional qed square

\newtheoremstyle{blacknumex}% Theorem style name
{5pt}% Space above
{5pt}% Space below
{\normalfont}% Body font
{} % Indent amount
{\small\bf\sffamily}% Theorem head font
{\;}% Punctuation after theorem head
{0.25em}% Space after theorem head
{\small\sffamily{\tiny\ensuremath{\blacksquare}}\nobreakspace\thmname{#1}\nobreakspace\thmnumber{\@ifnotempty{#1}{}\@upn{#2}}% Theorem text (e.g. Theorem 2.1)
\thmnote{\nobreakspace\the\thm@notefont\sffamily\bfseries---\nobreakspace#3.}}% Optional theorem note

\newtheoremstyle{blacknumbox} % Theorem style name
{0pt}% Space above
{0pt}% Space below
{\normalfont}% Body font
{}% Indent amount
{\small\bf\sffamily}% Theorem head font
{\;}% Punctuation after theorem head
{0.25em}% Space after theorem head
{\small\sffamily\thmname{#1}\nobreakspace\thmnumber{\@ifnotempty{#1}{}\@upn{#2}}% Theorem text (e.g. Theorem 2.1)
\thmnote{\nobreakspace\the\thm@notefont\sffamily\bfseries---\nobreakspace#3.}}% Optional theorem note

% Non-boxed/non-framed environments
\newtheoremstyle{ocrenum}% % Theorem style name
{5pt}% Space above
{5pt}% Space below
{\normalfont}% % Body font
{}% Indent amount
{\small\bf\sffamily\color{ocre}}% % Theorem head font
{\;}% Punctuation after theorem head
{0.25em}% Space after theorem head
{\small\sffamily\color{ocre}\thmname{#1}\nobreakspace\thmnumber{\@ifnotempty{#1}{}\@upn{#2}}% Theorem text (e.g. Theorem 2.1)
\thmnote{\nobreakspace\the\thm@notefont\sffamily\bfseries\color{black}---\nobreakspace#3.}} % Optional theorem note
\renewcommand{\qedsymbol}{$\blacksquare$}% Optional qed square
\makeatother



% Define el estilo texto theorem para cada tipo definido anteriormente
\newcounter{dummy} 
\numberwithin{dummy}{section}
\theoremstyle{ocrenumbox}
\newtheorem{theoremeT}[dummy]{Teorema}  % (Pedro: Theorem)
\newtheorem{problem}{Problema}[chapter]  % (Pedro: Problem)
\newtheorem{exerciseT}{Ejercicio}[chapter] % (Pedro: Exercise)
\theoremstyle{blacknumex}
\newtheorem{exampleT}{Ejemplo}[chapter] % (Pedro: Example)
\theoremstyle{blacknumbox}
\newtheorem{vocabulary}{Vocabulario}[chapter]  % (Pedro: Vocabulary)
\newtheorem{definitionT}{Definición}[section]  % (Pedro: Definition)
\newtheorem{corollaryT}[dummy]{Corolario}  % (Pedro: Corollary)
\theoremstyle{ocrenum}
\newtheorem{proposition}[dummy]{Proposición} % (Pedro: Proposition)


\usepackage[framemethod=default]{mdframed}



\newcommand{\intoo}[2]{\mathopen{]}#1\,;#2\mathclose{[}}
\newcommand{\ud}{\mathop{\mathrm{{}d}}\mathopen{}}
\newcommand{\intff}[2]{\mathopen{[}#1\,;#2\mathclose{]}}
\newtheorem{notation}{Notation}[chapter]


\mdfdefinestyle{exampledefault}{%
rightline=true,innerleftmargin=10,innerrightmargin=10,
frametitlerule=true,frametitlerulecolor=green,
frametitlebackgroundcolor=yellow,
frametitlerulewidth=2pt}


% Theorem box
\newmdenv[skipabove=7pt,
skipbelow=7pt,
backgroundcolor=black!5,
linecolor=ocre,
innerleftmargin=5pt,
innerrightmargin=5pt,
innertopmargin=10pt,%5pt
leftmargin=0cm,
rightmargin=0cm,
innerbottommargin=5pt]{tBox}

% Exercise box	  
\newmdenv[skipabove=7pt,
skipbelow=7pt,
rightline=false,
leftline=true,
topline=false,
bottomline=false,
backgroundcolor=ocre!10,
linecolor=ocre,
innerleftmargin=5pt,
innerrightmargin=5pt,
innertopmargin=10pt,%5pt
innerbottommargin=5pt,
leftmargin=0cm,
rightmargin=0cm,
linewidth=4pt]{eBox}	

% Definition box
\newmdenv[skipabove=7pt,
skipbelow=7pt,
rightline=false,
leftline=true,
topline=false,
bottomline=false,
linecolor=ocre,
innerleftmargin=5pt,
innerrightmargin=5pt,
innertopmargin=10pt,%0pt
leftmargin=0cm,
rightmargin=0cm,
linewidth=4pt,
innerbottommargin=0pt]{dBox}	

% Corollary box
\newmdenv[skipabove=7pt,
skipbelow=7pt,
rightline=false,
leftline=true,
topline=false,
bottomline=false,
linecolor=gray,
backgroundcolor=black!5,
innerleftmargin=5pt,
innerrightmargin=5pt,
innertopmargin=10pt,%5pt
leftmargin=0cm,
rightmargin=0cm,
linewidth=4pt,
innerbottommargin=5pt]{cBox}

% Crea un entorno para cada tipo de theorem y le asigna un estilo 
% con ayuda de las cajas coloreadas anteriores
\newenvironment{theorem}{\begin{tBox}\begin{theoremeT}}{\end{theoremeT}\end{tBox}}
\newenvironment{exercise}{\begin{eBox}\begin{exerciseT}}{\hfill{\color{ocre}\tiny\ensuremath{\blacksquare}}\end{exerciseT}\end{eBox}}				  
\newenvironment{definition}{\begin{dBox}\begin{definitionT}}{\end{definitionT}\end{dBox}}	
\newenvironment{example}{\begin{exampleT}}{\hfill{\tiny\ensuremath{\blacksquare}}\end{exampleT}}		
\newenvironment{corollary}{\begin{cBox}\begin{corollaryT}}{\end{corollaryT}\end{cBox}}	

%	ENVIRONMENT remark
\newenvironment{remark}{\par\vspace{10pt}\small 
% Espacio blanco vertical sobre la nota y tamaño de fuente menor
\begin{list}{}{
\leftmargin=35pt % Indentación sobre la izquierda
\rightmargin=25pt}\item\ignorespaces % Indentación sobre la derecha
\makebox[-2.5pt]{\begin{tikzpicture}[overlay]
\node[draw=ocre!60,line width=1pt,circle,fill=ocre!25,font=\sffamily\bfseries,inner sep=2pt,outer sep=0pt] at (-15pt,0pt){\textcolor{ocre}{N}}; \end{tikzpicture}} % R naranja en un círculo (Pedro)
\advance\baselineskip -1pt}{\end{list}\vskip5pt} 
% Espaciado de línea más estrecho y espacio en blanco después del comentario


\newenvironment{solutionExe}{\par\vspace{10pt}\small 
\begin{list}{}{
\leftmargin=35pt 
\rightmargin=25pt}\item\ignorespaces 
\makebox[-2.5pt]{\begin{tikzpicture}[overlay]
\node[draw=ocre!60,line width=1pt,circle,fill=ocre!25,font=\sffamily\bfseries,inner sep=2pt,outer sep=0pt] at (-15pt,0pt){\textcolor{ocre}{S}}; \end{tikzpicture}} 
\advance\baselineskip -1pt}{\end{list}\vskip5pt} 

\newenvironment{solutionExa}{\par\vspace{10pt}\small 
\begin{list}{}{
\leftmargin=35pt 
\rightmargin=25pt}\item\ignorespaces 
\makebox[-2.5pt]{\begin{tikzpicture}[overlay]
\node[draw=ocre!60,line width=1pt,circle,fill=ocre!55,font=\sffamily\bfseries,inner sep=2pt,outer sep=0pt] at (-15pt,0pt){\textcolor{ocre}{S}}; \end{tikzpicture}} 
\advance\baselineskip -1pt}{\end{list}\vskip5pt} 

\usepackage{tcolorbox}

\usetikzlibrary{trees}

\theoremstyle{ocrenum}
\newtheorem{solutionT}[dummy]{Solución}  % (Pedro: Corollary)
\newenvironment{solution}{\begin{cBox}\begin{solutionT}}{\end{solutionT}\end{cBox}}	


\newcommand{\tcolorboxsolucion}[2]{%
\begin{tcolorbox}[colback=green!5!white,colframe=green!75!black,title=#1] 
 #2
 %\tcblower  % pone una línea discontinua
\end{tcolorbox}
}% final definición comando

\newtcbox{\mybox}[1][green]{on line,
arc=0pt,outer arc=0pt,colback=#1!10!white,colframe=#1!50!black, boxsep=0pt,left=1pt,right=1pt,top=2pt,bottom=2pt, boxrule=0pt,bottomrule=1pt,toprule=1pt}



\mdfdefinestyle{exampledefault}{%
rightline=true,innerleftmargin=10,innerrightmargin=10,
frametitlerule=true,frametitlerulecolor=green,
frametitlebackgroundcolor=yellow,
frametitlerulewidth=2pt}





\newcommand{\betheorem}{\begin{theorem}}
\newcommand{\eetheorem}{\end{theorem}}
\newcommand{\bedefinition}{\begin{definition}}
\newcommand{\eedefinition}{\end{definition}}

\newcommand{\beremark}{\begin{remark}}
\newcommand{\eeremark}{\end{remark}}
\newcommand{\beexercise}{\begin{exercise}}
\newcommand{\eeexercise}{\end{exercise}}
\newcommand{\beexample}{\begin{example}}
\newcommand{\eeexample}{\end{example}}
\newcommand{\becorollary}{\begin{corollary}}
\newcommand{\eecorollary}{\end{corollary}}


\newcommand{\besolutionExe}{\begin{solutionExe}}
\newcommand{\eesolutionExe}{\end{solutionExe}}
\newcommand{\besolutionExa}{\begin{solutionExa}}
\newcommand{\eesolutionExa}{\end{solutionExa}}


%%%%%%%%


% Caja Salida Markdown
\newmdenv[skipabove=7pt,
skipbelow=7pt,
rightline=false,
leftline=true,
topline=false,
bottomline=false,
backgroundcolor=GreenYellow!10,
linecolor=GreenYellow!80,
innerleftmargin=5pt,
innerrightmargin=5pt,
innertopmargin=10pt,%5pt
innerbottommargin=5pt,
leftmargin=0cm,
rightmargin=0cm,
linewidth=4pt]{mBox}	

%% RMarkdown
\newenvironment{markdownsal}{\begin{mBox}}{\end{mBox}}	

\newcommand{\bmarkdownsal}{\begin{markdownsal}}
\newcommand{\emarkdownsal}{\end{markdownsal}}


\usepackage{array}
\usepackage{multirow}
\usepackage{wrapfig}
\usepackage{colortbl}
\usepackage{pdflscape}
\usepackage{tabu}
\usepackage{threeparttable}
\usepackage{subfig} %new
%\usepackage{booktabs,dcolumn,rotating,thumbpdf,longtable}
\usepackage{dcolumn,rotating}  %new
\usepackage[graphicx]{realboxes} %new de: https://stackoverflow.com/questions/51633434/prevent-pagebreak-in-kableextra-landscape-table

%define el interlineado vertical
%\renewcommand{\baselinestretch}{1.5}

%define etiqueta para las Tablas o Cuadros
%\renewcommand\spanishtablename{Tabla}

%%\bibliographystyle{plain} %new no necesario


%%%%%%%%%%%% PARA USO CON biblatex
% \DefineBibliographyStrings{english}{%
%   backrefpage = {ver pag.\adddot},%
%   backrefpages = {ver pags.\adddot}%
% }

% \DefineBibliographyStrings{spanish}{%
%   backrefpage = {ver pag.\adddot},%
%   backrefpages = {ver pags.\adddot}%
% }
% 
% \DeclareFieldFormat{pagerefformat}{\mkbibparens{{\color{red}\mkbibemph{#1}}}}
% \renewbibmacro*{pageref}{%
%   \iflistundef{pageref}
%     {}
%     {\printtext[pagerefformat]{%
%        \ifnumgreater{\value{pageref}}{1}
%          {\bibstring{backrefpages}\ppspace}
%          {\bibstring{backrefpage}\ppspace}%
%        \printlist[pageref][-\value{listtotal}]{pageref}}}}
% 
%%% de kableExtra
\usepackage{booktabs}
\usepackage{longtable}
%\usepackage{array}
%\usepackage{multirow}
%\usepackage{wrapfig}
%\usepackage{float}
%\usepackage{colortbl}
%\usepackage{pdflscape}
%\usepackage{tabu}
%\usepackage{threeparttable}
\usepackage{threeparttablex}
\usepackage[normalem]{ulem}
\usepackage{makecell}
%\usepackage{xcolor}

%%%%%%% fin: latex_preambulo.tex PLLC


\begin{document}

\bibliographystyle{flexbib}



\raggedbottom

\ifdefined\ifprincipal
\else
\setlength{\parindent}{1em}
\pagestyle{fancy}
\setcounter{tocdepth}{4}
\tableofcontents

\fi

\ifdefined\ifdoblecara
\fancyhead{}{}
\fancyhead[LE,RO]{\scriptsize\rightmark}
\fancyfoot[LO,RE]{\scriptsize\slshape \leftmark}
\fancyfoot[C]{}
\fancyfoot[LE,RO]{\footnotesize\thepage}
\else
\fancyhead{}{}
\fancyhead[RO]{\scriptsize\rightmark}
\fancyfoot[LO]{\scriptsize\slshape \leftmark}
\fancyfoot[C]{}
\fancyfoot[RO]{\footnotesize\thepage}
\fi

\renewcommand{\headrulewidth}{0.4pt}
\renewcommand{\footrulewidth}{0.4pt}

\hypertarget{construcciuxf3n-de-la-base-de-datos}{%
\chapter{Construcción de la base de
datos}\label{construcciuxf3n-de-la-base-de-datos}}

El primer paso a la hora de construir cualquier modelo de predicción es
disponer de datos adecuados que permitan explicar correctamente el
fenómeno en estudio, en este caso los incendios forestales en Andalucía.
Con este fin, se ha llevado a cabo un extenso estudio previo del dominio
del problema para conocer qué variables pueden ser relevantes de cara a
la predicción de incendios forestales, analizando estudios similares
realizados anteriormente así como otras fuentes relativas a la ecología
del fuego, que nos permitiesen conocer el efecto que cabría esperar de
estas variables.

Se ha querido adoptar un enfoque dinámico, es decir, el objetivo no es
construir un modelo estacionario que nos indique si una determinada zona
se verá afectada por un incendio forestal a lo largo de un amplio
periodo temporal, si no que se pretende ser capaz de predecir si un
determinado punto del territorio andaluz se verá afectado por un
incendio forestal en un momento concreto, en base a las covariables
correspondientes a ese lugar en ese momento. Es decir, se considera no
solo la dimensión espacial de los datos si no también la temporal, al
mayor nivel de desagregación disponible. Este es un enfoque mucho menos
explorado, debido fundamentalmente a dos factores:

\begin{enumerate}
\def\labelenumi{\arabic{enumi}.}
\tightlist
\item
  La dificultad de disponer de información fiable y de calidad
  desagregada espacio-temporalmente
\item
  La dificultad de trabajar con datos de estas características de cara
  al análisis y principalmente a la modelización, ya que son datos
  correlados en el tiempo y en el espacio.
\end{enumerate}

Queda claro, por tanto, que se trata de un problema complejo que
requiere de simplificaciones para poder ser abordado, más aun dadas las
limitaciones en los recursos computacionales disponibles y la enorme
cantidad de de datos que se están considerando y que requieren de un
procesamiento sumamente costoso desde un punto de vista computacional.

Por todo ello, esta sección es probablemente la de mayor importancia y
dificultad de todo el trabajo, ya que implica la toma de decisiones que
serán determinantes de cara al correcto desempeño de los modelos que se
construirán más adelante, requiere de un vasto conocimiento del problema
que permita un enfoque adecuado que haga posible la consecución de los
objetivos que se esperan conseguir, necesita del uso de técnicas
específicas de procesamiento de datos espaciales que no han sido
tratadas durante el grado y se ve fuertemente limitada por los escasos
recursos computacionales disponibles.

\hypertarget{determinaciuxf3n-del-marco-del-estudio}{%
\section{Determinación del marco del
estudio}\label{determinaciuxf3n-del-marco-del-estudio}}

El primer paso ha sido limitar el área y la franja temporal que abarcará
el estudio. Para ello, ha sido necesario basarse principalmente en la
disponibilidad y consistencia de la información requerida para el
proyecto y en las limitaciones computacionales impuestas por el equipo
disponible.

En cuanto a la disponibilidad de información, hay que diferenciar entre
la información de incendios forestales y la información de variables que
permitan explicar este fenómeno considerando la mayor desagregación
espacial y temporal posible.

\hypertarget{incendios-forestales}{%
\subsection{Incendios forestales}\label{incendios-forestales}}

En lo referente a los datos sobre incendios forestales cabe mencionar
que España cuenta con una de las mayores y más completas bases de datos
sobre incendios forestales a nivel europeo. Se trata de la Estadística
General de Incendios Forestales (EGIF), que en su versión definitiva
actualmente contiene toda la información que se recoge en cada parte de
incendio forestal que ha tenido lugar en España desde 1983 hasta 2015,
incluyendo su información espacial con sus coordenadas de origen. Se ha
explorado extensamente el uso de esta base de datos para el proyecto,
dada su exhaustividad y completitud. Sin embargo, lamentablemente no ha
sido posible en este caso incorporarla al trabajo por diversas razones.

La principal de ellas fue que hasta marzo de 2024 la base de datos de la
EGIF solo se encontraba disponible en el Catálogo de Datos del Gobierno
de España en formato TURTLE y esto conllevó numerosas dificultades. Se
exploraron distintas librerías de R (y alguna de Python) para el manejo
de datos en este formato como RDFlib. Sin embargo, al tratarse de una
base de datos de un tamaño considerable (aproximadamente 1GB y con más
de una decena de millones de tripletas), esta librería no era
suficientemente eficiente para poder realizar consultas en un tiempo
razonable al conjunto de datos. Tras explorar otras alternativas, se
valoró la posibilidad de usar un triplestore, es decir, una base de
datos especialmente diseñada para el almacenamiento y recuperación de
tripletas a través de consultas semánticas. En este caso se usó Apache
Jena Fuseki, ya que cuenta con una interfaz que facilita su uso. Sin
embargo, aunque esto supuso una mejora considerable en la eficiencia y
permitió realizar consultas sencillas a la base de datos, en este caso
fue la complejidad del gráfico de datos (ontología) y la escasa
documentación disponible sobre esta, la impidió que pudiese realizar las
consultas más complejas que requería para llevar a cabo el proyecto.
Además, se debeb tener en cuenta que se trata de una base de datos muy
heterogénea y con numerosos datos faltantes debida su naturaleza, por lo
que requiere de un preprocesamiento que probablemente será complicado y
costoso en tiempo y en recursos computacionales. Al no disponer de
ninguno de estos, finalmente se optó por buscar una alternativa más
abarcable dada las limitaciones con las que cuenta un Trabajo de Fin de
Estudios, aunque queda abierta la posibilidad de explorar esta base de
datos en futuros estudios, la cual aportar nuevas dimensiones al estudio
de los incendios forestales en España gracias a la enorme cantidad de
información que ofrece.

TURTLE es una sintaxis para RDF con una sintaxis compatible con SPARQL.
RDF (Resource Description Framework) es un estándar de semántica web
utilizado para el intercambiar de datos en la Web.

Ante esta situación, la solución planteada fue limitar el área en
estudio a la Comunidad Autónoma de Andalucía, aprovechando la enorme
disponibilidad de información medioambiental que ofrece la Red de
Información Ambiental de Andalucía (REDIAM). En particular, se emplea la
cartografía generada por la REDIAM sobre las áreas recorridas por los
incendios forestales entre 1975 y 2022. Esta contiene los perímetros de
incendios forestales mayores de 100 ha en Andalucía obtenidos a partir
de imágenes de satélite y datos de campo. Se trata por tanto de una
información que no es exhaustiva, pues los incendios con una extensión
inferior a 100ha no han sido considerados. Sin embargo, frente a no
disponer de otra información operativa de mayor calidad, se utilizará
esta teniendo en cuenta que tendrá un efecto sobre las conclusiones que
se puedan sacar de los modelos que se construyan.

\hypertarget{variables-predictoras}{%
\subsection{Variables predictoras}\label{variables-predictoras}}

Una vez limitada la extensión territorial del estudio el siguiente paso
era acotar la franja temporal que abarcaría el estudio en base a la
disponibilidad de datos adecuados para explicar el fenómeno en cuestión
desagregados espacial y temporalmente.

Los incendios forestales son un proceso sumamente complejo, en el que
actúan numerosos factores de muy distinta índole (\ldots). Además,
dentro de un incendio forestal se pueden distinguir distintas fases que
presentan características muy diversas y sobre las que actúan distintos
agentes: ignición, propagación y extinción. Dada la información sobre
incendios forestales disponible, se está obligado a adoptar un enfoque
global, pues no se dispone de los puntos de ignición u origen de los
incendios forestales. El enfoque será, por tanto, intentar predecir si
una determinada localización se verá afectada por un incendio forestal
(de más de 100 ha) en un momento concreto.

Además, es importante tener en cuenta que existen factores estructurales
que tienen una influencia directa sobre los regímenes de incendios
forestales como son las tendencias de uso y explotación de los bosques,
la presencia de interfaz urbano forestal, los tipos y técnicas de
agricultura que se llevan a cabo, la presencia e intensidad del
pastoreo, los cambios en los usos de suelo e incluso conductas sociales
y tendencias demográficas diversas. Se trata de variables que cambian a
lo largo de periodos relativamente largos de tiempo y que muy
difícilmente pueden ser incluidos en los modelos, dada la falta de datos
sobre ellas, así como su carácter transversal. Por ello, se ha
considerado conveniente no extender en exceso el periodo de estudio,
reconocida la imposibilidad de incluir en el modelo todas las variables
que tienen un impacto relevante en la aparición de incendios y que son
cambiantes en el tiempo.

Todo ello hace necesario que el conjunto de datos utilizado contenga
información sobre todas las dimensiones (o al menos las principales) que
influyen en cualquiera de las fases de un incendio forestal. Es decir,
se deben incluir la dimensión antropogénica, la demográfica, la
hidrográficas, la topográfica, la meteorológica y la vegetación. Es
importante recalcar que siempre se hace referencia a datos geoespaciales
pues debe ser la información relativa al lugar (y al momento) del
incendio, con la dificultad posterior que esto supondrá.

Por último, es importante diferenciar entre características que se
considerarán estructurales (y por tanto invariantes a lo largo del
periodo de estudio) y aquellas que se considerarán variables en el
tiempo. Dentro de las primeras se encuentran todas las características
relacionadas con la topografía del terreno, las infraestructuras y los
usos del suelo, como por ejemplo el modelo de elevaciones, la
distribución de asentamientos de población, la red de carreteras y el
uso de suelo. Todas las demás variables de carácter demográfico,
meteorológico o de vegetación se considerarán, por tanto, desagregadas
temporalmente.

En base a todo lo mencionado y a la disponibilidad de información de
calidad de las categorías comentadas, se ha decidido limitar la franja
temporal del estudio a 20 años que van de 2002 a 2022, ambos inclusive.

\hypertarget{fuentes-de-datos}{%
\section{Fuentes de datos}\label{fuentes-de-datos}}

Como se ha comentado en la sección anterior, los datos sobre los
incendios forestales se han obtenido de los perímetros de incendios
forestales mayores de 100 ha en Andalucía entre 1975 y 2020 disponibles
la REDIAM. De cada incendio registrado se dispone de su fecha de inicio,
del área recorrida por el fuego y del municipio en el que originó, así
como de otras variables que dependen del año de la campaña y que no son
relevantes de cara a nuestro estudio.

Tomando como base estudios similares (\ldots) y partiendo de las 6
categorías ya mencionadas se han recopilado 23 conjuntos de datos de
distinto tipo que se usarán para explicar y predecir los incendios
forestales en Andalucía. Estos conjuntos se recogen en la Tabla
\ref{tab:fuentes}, donde también se indica la fuente de la que ha sido
obtenido cada uno de ellos, el tipo de datos que contiene (indicando su
resolución en el caso de los datos ráster) y la frecuencia de las
observaciones (o resolución temporal) en el caso de las variables
temporales.

Es relevante la heterogeneidad de los datos recopilados, pues se dispone
tanto de datos tabulares como de datos espaciales y dentro de estos
últimos de datos vectoriales y datos ráster, con distintas resoluciones,
distintas frecuencias y distintos sistemas de referencia de coordenadas.
Esto hará que el procesamiento de estos datos hasta obtener datos
adecuados para el análisis estadístico sea costoso y que deban
utilizarse técnicas específicas de geocumputación.

Cabe también mencionar que se ha optado por el uso de datos
meteorológicos basados en modelos y en observaciones satelitares, en
lugar del uso de datos provenientes de estaciones meteorológicas. Si
bien la información de estaciones meteorológica puede ser más precisa,
la dificultad de disponer de datos consistentes y continuos en el tiempo
a lo largo del periodo de estudio de las variables meteorológicas
seleccionadas ha hecho que este enfoque no sea viable. En esta dirección
se ha explorado la API de la AEMET y algunos paquetes de R como
\texttt{climate}, sin llegar a resultados satisfactorios. Por otro lado,
el paquete \texttt{nasapower} permite la descarga de una gran cantidad
de variables meteorológicas con frecuencia diaria y con una resolución
de aproximadamente \(0.5 \times 0.625\) grados de latitud y longitud
(unos 50km). Si bien es cierto que no es lo ideal, es la única opción
que se ha considerado viable y de cara a la construcción de unos
primeros modelos aproximativos podría ser suficiente. Si quisiese
extenderse el estudio, sería conveniente profundizar en la búsqueda de
alternativas que permitan obtener información meteorológica de una mayor
calidad.

\begin{table}
\begin{threeparttable}[]
\resizebox{\textwidth}{!}{
\begin{tabular}{lllll}
\hline
\textbf{Categoría}     & \textbf{Dato}          & \textbf{Fuente}    & \textbf{Tipo de dato}       & \textbf{Frecuencia} \\ \hline
\multirow{4}{*}{Topográficas}   & Altitud                                                        & DERA\tnote{a}  & TIFF (100m)        & -          \\
                                & Orientación                                                    & REDIAM\tnote{b}     & TIFF (100m)        & -          \\
                                & Pendiente                                                      & REDIAM     & TIFF (100m)        & -          \\
                                & Curvatura                                                      & REDIAM     & TIFF (100m)        & -          \\ \hline
Vegetación                      & NDVI                                                           & REDIAM     & TIFF (250m)        & Mensual    \\ \hline
\multirow{7}{*}{Antropogénicas} & Uso de suelo                                                   & DERA       & Shapefile          & -          \\
                                & Red de carreteras                                              & DERA       & Shapefile          & -          \\
                                & Red de ferrocarril                                             & DERA       & Shapefile          & -          \\
                                & Línea eléctrica                                                & DERA       & Shapefile          & -          \\
                                & Espacio protegido                                              & DERA       & Shapefile          & -          \\
                                & Senderos / Vías Verde / Carriles Bici                          & DERA       & Shapefile          & -          \\
                                & Caminos / Vías Pecuarias                                       & DERA       & Shapefile          & -          \\ \hline
Demográficas                    & Población del municipio                                        & IECA\tnote{c}       & csv                & Anual      \\ \hline
Hidrográficas                   & Principales Ríos                                               & MAGRAMA\tnote{d}    & Shapefile          & -          \\ \hline
\multirow{6}{*}{Meteorológicas} & Precipitación (mm/day)                                         & NASA POWER\tnote{e} & df (0.5º x 0.625º) & Diaria     \\
                                & Temperatura a 2m sobre la superficie (º)                       & NASA POWER & df (0.5º x 0.625º) & Diaria     \\
                                & Humedad del suelo (\%)                                         & NASA POWER & df (0.5º x 0.625º) & Diaria     \\
                                & Dirección del viento a 10 metros sobre la superficie terrestre(º) & NASA POWER & df (0.5º x 0.625º) & Diaria     \\
                                & Humedad relativa a 2m sobre la superficie (\%)                 & NASA POWER & df (0.5º x 0.625º) & Diaria     \\
                                & Cantidad de precipitaciones (mm/day)                           & NASA POWER & dfdf (0.5º x 0.625º) & Diaria     \\ \hline
\footnotesize Fuente: Elaboración propia
\end{tabular}}
\begin{tablenotes}
\raggedright
\item[a] {\footnotesize \href{https://www.juntadeandalucia.es/institutodeestadisticaycartografia/dega/datos-espaciales-de-referencia-de-andalucia-dera/descarga-de-informacion}{Datos Espaciales de Referencia de Andalucía (DERA)}}
\item[b] {\footnotesize \href{https://portalrediam.cica.es/descargas?path=%2F}{Descargas Rediam}}%2F}{Descargas Rediam}}
\item[c] {\footnotesize \href{https://www.juntadeandalucia.es/institutodeestadisticaycartografia/dega/}{Instituto de Estadística y Cartografía de Andalucía (IECA)}}
\item[d] {\footnotesize Ministerio de Agricultura, Alimentación y Medio Ambiente (MAGRAMA)}
\item[e] {\footnotesize \href{https://power.larc.nasa.gov/#resources}{NASA Prediction Of Worlwide Energy Resources (NASA POWER)}}
\end{tablenotes}
\caption{Datos brutos}
\label{tab:fuentes}
\end{threeparttable}
\end{table}

\hypertarget{procesamiento-de-los-datos}{%
\section{Procesamiento de los datos}\label{procesamiento-de-los-datos}}

Una vez se dispone de todos los conjuntos de datos que se usarán en el
estudio, el siguiente paso será combinarlos de manera adecuada y
transformarlos a un formato apto para el análisis estadístico y la
construcción de modelos predictivos, es decir, a un data frame. Dado que
el objetivo que se persigue es predecir si, dada unas condiciones
meteorológicas concretas en un momento dado, un punto del territorio
andaluz se verá afectado o no por un incendio forestal, será necesario
disponer de una cantidad suficiente de muestras negativas y positivas
distribuidas espacial y temporalmente que tengan asociadas las variables
explicativas correspondientes.

Intuitivamente, las muestras positivas serán aquellas observaciones
(puntos definidos en el tiempo y en el espacio) dentro del marco
espacio-temporal del estudio en las que se ha detectado un incendio
forestal en el día de la observación. Es decir, son observaciones dentro
de los polígonos de incendios el día que estos se han producido. Por
tanto, las muestras negativas serán observaciones dentro del marco
espacio-temporal definido en las que no se ha detectado un incendio
forestal.

A continuación se detalla el proceso seguido para generar el conjunto de
datos depurado sobre el que desarrollar el estudio a partir de los
distintos conjuntos de datos en bruto:

\begin{enumerate}
\def\labelenumi{\arabic{enumi}.}
\item
  Generación de una muestra balanceada de casos positivos y negativos.
\item
  Asignación a cada observación los valores correspondientes a ese día y
  a esa localización concreta de todas las variables predictoras.
\item
  Depuración de la muestra generada, elimininando valores perdidos y
  ajustando adecuadamente los tipos de las variables.
\end{enumerate}

\hypertarget{depuraciuxf3n-de-los-datos}{%
\section{Depuración de los datos}\label{depuraciuxf3n-de-los-datos}}

Para la información sobre incendios forestales en Andalucía se ha
empleado la cartografía generada por la Red de Información Ambiental de
Andalucía (REDIAM) sobre áreas recorridas por los incendios forestales
en Andalucía entre 1975 y 2022. Se acota la el periodo de estudio,
cambios estructurales en los regímenes de incendios y disponibilidad de
la información

\bibliography{bib/library.bib,bib/paquetes.bib}


%


\end{document}
