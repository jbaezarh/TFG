\documentclass[12pt,a4paper,]{book}
\def\ifdoblecara{} %% set to true
\def\ifprincipal{} %% set to true
\let\ifprincipal\undefined %% set to false
\def\ifcitapandoc{} %% set to true
\let\ifcitapandoc\undefined %% set to false
\usepackage{lmodern}
% sin fontmathfamily
\usepackage{amssymb,amsmath}
\usepackage{ifxetex,ifluatex}
%\usepackage{fixltx2e} % provides \textsubscript %PLLC
\ifnum 0\ifxetex 1\fi\ifluatex 1\fi=0 % if pdftex
  \usepackage[T1]{fontenc}
  \usepackage[utf8]{inputenc}
\else % if luatex or xelatex
  \ifxetex
    \usepackage{mathspec}
  \else
    \usepackage{fontspec}
  \fi
  \defaultfontfeatures{Ligatures=TeX,Scale=MatchLowercase}
\fi
% use upquote if available, for straight quotes in verbatim environments
\IfFileExists{upquote.sty}{\usepackage{upquote}}{}
% use microtype if available
\IfFileExists{microtype.sty}{%
\usepackage{microtype}
\UseMicrotypeSet[protrusion]{basicmath} % disable protrusion for tt fonts
}{}
\usepackage[margin = 2.5cm]{geometry}
\usepackage{hyperref}
\hypersetup{unicode=true,
            pdfauthor={Nombre Completo Autor},
              pdfborder={0 0 0},
              breaklinks=true}
\urlstyle{same}  % don't use monospace font for urls
%
\usepackage[usenames,dvipsnames]{xcolor}  %new PLLC
\usepackage{graphicx,grffile}
\makeatletter
\def\maxwidth{\ifdim\Gin@nat@width>\linewidth\linewidth\else\Gin@nat@width\fi}
\def\maxheight{\ifdim\Gin@nat@height>\textheight\textheight\else\Gin@nat@height\fi}
\makeatother
% Scale images if necessary, so that they will not overflow the page
% margins by default, and it is still possible to overwrite the defaults
% using explicit options in \includegraphics[width, height, ...]{}
\setkeys{Gin}{width=\maxwidth,height=\maxheight,keepaspectratio}
\IfFileExists{parskip.sty}{%
\usepackage{parskip}
}{% else
\setlength{\parindent}{0pt}
\setlength{\parskip}{6pt plus 2pt minus 1pt}
}
\setlength{\emergencystretch}{3em}  % prevent overfull lines
\providecommand{\tightlist}{%
  \setlength{\itemsep}{0pt}\setlength{\parskip}{0pt}}
\setcounter{secnumdepth}{5}
% Redefines (sub)paragraphs to behave more like sections
\ifx\paragraph\undefined\else
\let\oldparagraph\paragraph
\renewcommand{\paragraph}[1]{\oldparagraph{#1}\mbox{}}
\fi
\ifx\subparagraph\undefined\else
\let\oldsubparagraph\subparagraph
\renewcommand{\subparagraph}[1]{\oldsubparagraph{#1}\mbox{}}
\fi

%%% Use protect on footnotes to avoid problems with footnotes in titles
\let\rmarkdownfootnote\footnote%
\def\footnote{\protect\rmarkdownfootnote}


  \title{}
    \author{Nombre Completo Autor}
      \date{18/11/2021}


%%%%%%% inicio: latex_preambulo.tex PLLC


%% UTILIZA CODIFICACIÓN UTF-8
%% MODIFICARLO CONVENIENTEMENTE PARA USARLO CON OTRAS CODIFICACIONES


%\usepackage[spanish,es-nodecimaldot,es-noshorthands]{babel}
\usepackage[spanish,es-nodecimaldot,es-noshorthands,es-tabla]{babel}
% Ver: es-tabla (en: https://osl.ugr.es/CTAN/macros/latex/contrib/babel-contrib/spanish/spanish.pdf)
% es-tabla (en: https://tex.stackexchange.com/questions/80443/change-the-word-table-in-table-captions)
\usepackage[spanish, plain, datebegin,sortcompress,nocomment,
noabstract]{flexbib}
 
\usepackage{float}
\usepackage{placeins}
\usepackage{fancyhdr}
% Solucion: ! LaTeX Error: Command \counterwithout already defined.
% https://tex.stackexchange.com/questions/425600/latex-error-command-counterwithout-already-defined
\let\counterwithout\relax
\let\counterwithin\relax
\usepackage{chngcntr}
%\usepackage{microtype}  %antes en template PLLC
\usepackage[utf8]{inputenc}
\usepackage[T1]{fontenc} % Usa codificación 8-bit que tiene 256 glyphs

%\usepackage[dvipsnames]{xcolor}
%\usepackage[usenames,dvipsnames]{xcolor}  %new
\usepackage{pdfpages}
%\usepackage{natbib}




% Para portada: latex_paginatitulo_mod_ST02.tex (inicio)
\usepackage{tikz}
\usepackage{epigraph}
\renewcommand\epigraphflush{flushright}
\renewcommand\epigraphsize{\normalsize}
\setlength\epigraphwidth{0.7\textwidth}

\definecolor{titlepagecolor}{cmyk}{1,.60,0,.40}

%\DeclareFixedFont{\titlefont}{T1}{ppl}{b}{it}{0.5in}

% \makeatletter
% \def\printauthor{%
%     {\large \@author}}
% \makeatother
% \author{%
%     Author 1 name \\
%     Department name \\
%     \texttt{email1@example.com}\vspace{20pt} \\
%     Author 2 name \\
%     Department name \\
%     \texttt{email2@example.com}
%     }

% The following code is borrowed from: https://tex.stackexchange.com/a/86310/10898

\newcommand\titlepagedecoration{%
\begin{tikzpicture}[remember picture,overlay,shorten >= -10pt]

\coordinate (aux1) at ([yshift=-15pt]current page.north east);
\coordinate (aux2) at ([yshift=-410pt]current page.north east);
\coordinate (aux3) at ([xshift=-4.5cm]current page.north east);
\coordinate (aux4) at ([yshift=-150pt]current page.north east);

\begin{scope}[titlepagecolor!40,line width=12pt,rounded corners=12pt]
\draw
  (aux1) -- coordinate (a)
  ++(225:5) --
  ++(-45:5.1) coordinate (b);
\draw[shorten <= -10pt]
  (aux3) --
  (a) --
  (aux1);
\draw[opacity=0.6,titlepagecolor,shorten <= -10pt]
  (b) --
  ++(225:2.2) --
  ++(-45:2.2);
\end{scope}
\draw[titlepagecolor,line width=8pt,rounded corners=8pt,shorten <= -10pt]
  (aux4) --
  ++(225:0.8) --
  ++(-45:0.8);
\begin{scope}[titlepagecolor!70,line width=6pt,rounded corners=8pt]
\draw[shorten <= -10pt]
  (aux2) --
  ++(225:3) coordinate[pos=0.45] (c) --
  ++(-45:3.1);
\draw
  (aux2) --
  (c) --
  ++(135:2.5) --
  ++(45:2.5) --
  ++(-45:2.5) coordinate[pos=0.3] (d);   
\draw 
  (d) -- +(45:1);
\end{scope}
\end{tikzpicture}%
}

% Para portada: latex_paginatitulo_mod_ST02.tex (fin)

% Para portada: latex_paginatitulo_mod_OV01.tex (inicio)
\usepackage{cpimod}
% Para portada: latex_paginatitulo_mod_OV01.tex (fin)

% Para portada: latex_paginatitulo_mod_OV03.tex (inicio)
\usepackage{KTHEEtitlepage}
% Para portada: latex_paginatitulo_mod_OV03.tex (fin)

\renewcommand{\contentsname}{Índice}
\renewcommand{\listfigurename}{Índice de figuras}
\renewcommand{\listtablename}{Índice de tablas}
\newcommand{\bcols}{}
\newcommand{\ecols}{}
\newcommand{\bcol}[1]{\begin{minipage}{#1\linewidth}}
\newcommand{\ecol}{\end{minipage}}
\newcommand{\balertblock}[1]{\begin{alertblock}{#1}}
\newcommand{\ealertblock}{\end{alertblock}}
\newcommand{\bitemize}{\begin{itemize}}
\newcommand{\eitemize}{\end{itemize}}
\newcommand{\benumerate}{\begin{enumerate}}
\newcommand{\eenumerate}{\end{enumerate}}
\newcommand{\saltopagina}{\newpage}
\newcommand{\bcenter}{\begin{center}}
\newcommand{\ecenter}{\end{center}}
\newcommand{\beproof}{\begin{proof}} %new
\newcommand{\eeproof}{\end{proof}} %new
%De: https://texblog.org/2007/11/07/headerfooter-in-latex-with-fancyhdr/
% \fancyhead
% E: Even page
% O: Odd page
% L: Left field
% C: Center field
% R: Right field
% H: Header
% F: Footer
%\fancyhead[CO,CE]{Resultados}

%OPCION 1
% \fancyhead[LE,RO]{\slshape \rightmark}
% \fancyhead[LO,RE]{\slshape \leftmark}
% \fancyfoot[C]{\thepage}
% \renewcommand{\headrulewidth}{0.4pt}
% \renewcommand{\footrulewidth}{0pt}

%OPCION 2
% \fancyhead[LE,RO]{\slshape \rightmark}
% \fancyfoot[LO,RE]{\slshape \leftmark}
% \fancyfoot[LE,RO]{\thepage}
% \renewcommand{\headrulewidth}{0.4pt}
% \renewcommand{\footrulewidth}{0.4pt}
%%%%%%%%%%
\usepackage{calc,amsfonts}
% Elimina la cabecera de páginas impares vacías al finalizar los capítulos
\usepackage{emptypage}
\makeatletter

%\definecolor{ocre}{RGB}{25,25,243} % Define el color azul (naranja) usado para resaltar algunas salidas
\definecolor{ocre}{RGB}{0,0,0} % Define el color a negro (aparece en los teoremas

%\usepackage{calc} 


%era if(csl-refs) con dolares
% metodobib: true


\usepackage{lipsum}

%\usepackage{tikz} % Requerido para dibujar formas personalizadas

%\usepackage{amsmath,amsthm,amssymb,amsfonts}
\usepackage{amsthm}


% Boxed/framed environments
\newtheoremstyle{ocrenumbox}% % Theorem style name
{0pt}% Space above
{0pt}% Space below
{\normalfont}% % Body font
{}% Indent amount
{\small\bf\sffamily\color{ocre}}% % Theorem head font
{\;}% Punctuation after theorem head
{0.25em}% Space after theorem head
{\small\sffamily\color{ocre}\thmname{#1}\nobreakspace\thmnumber{\@ifnotempty{#1}{}\@upn{#2}}% Theorem text (e.g. Theorem 2.1)
\thmnote{\nobreakspace\the\thm@notefont\sffamily\bfseries\color{black}---\nobreakspace#3.}} % Optional theorem note
\renewcommand{\qedsymbol}{$\blacksquare$}% Optional qed square

\newtheoremstyle{blacknumex}% Theorem style name
{5pt}% Space above
{5pt}% Space below
{\normalfont}% Body font
{} % Indent amount
{\small\bf\sffamily}% Theorem head font
{\;}% Punctuation after theorem head
{0.25em}% Space after theorem head
{\small\sffamily{\tiny\ensuremath{\blacksquare}}\nobreakspace\thmname{#1}\nobreakspace\thmnumber{\@ifnotempty{#1}{}\@upn{#2}}% Theorem text (e.g. Theorem 2.1)
\thmnote{\nobreakspace\the\thm@notefont\sffamily\bfseries---\nobreakspace#3.}}% Optional theorem note

\newtheoremstyle{blacknumbox} % Theorem style name
{0pt}% Space above
{0pt}% Space below
{\normalfont}% Body font
{}% Indent amount
{\small\bf\sffamily}% Theorem head font
{\;}% Punctuation after theorem head
{0.25em}% Space after theorem head
{\small\sffamily\thmname{#1}\nobreakspace\thmnumber{\@ifnotempty{#1}{}\@upn{#2}}% Theorem text (e.g. Theorem 2.1)
\thmnote{\nobreakspace\the\thm@notefont\sffamily\bfseries---\nobreakspace#3.}}% Optional theorem note

% Non-boxed/non-framed environments
\newtheoremstyle{ocrenum}% % Theorem style name
{5pt}% Space above
{5pt}% Space below
{\normalfont}% % Body font
{}% Indent amount
{\small\bf\sffamily\color{ocre}}% % Theorem head font
{\;}% Punctuation after theorem head
{0.25em}% Space after theorem head
{\small\sffamily\color{ocre}\thmname{#1}\nobreakspace\thmnumber{\@ifnotempty{#1}{}\@upn{#2}}% Theorem text (e.g. Theorem 2.1)
\thmnote{\nobreakspace\the\thm@notefont\sffamily\bfseries\color{black}---\nobreakspace#3.}} % Optional theorem note
\renewcommand{\qedsymbol}{$\blacksquare$}% Optional qed square
\makeatother



% Define el estilo texto theorem para cada tipo definido anteriormente
\newcounter{dummy} 
\numberwithin{dummy}{section}
\theoremstyle{ocrenumbox}
\newtheorem{theoremeT}[dummy]{Teorema}  % (Pedro: Theorem)
\newtheorem{problem}{Problema}[chapter]  % (Pedro: Problem)
\newtheorem{exerciseT}{Ejercicio}[chapter] % (Pedro: Exercise)
\theoremstyle{blacknumex}
\newtheorem{exampleT}{Ejemplo}[chapter] % (Pedro: Example)
\theoremstyle{blacknumbox}
\newtheorem{vocabulary}{Vocabulario}[chapter]  % (Pedro: Vocabulary)
\newtheorem{definitionT}{Definición}[section]  % (Pedro: Definition)
\newtheorem{corollaryT}[dummy]{Corolario}  % (Pedro: Corollary)
\theoremstyle{ocrenum}
\newtheorem{proposition}[dummy]{Proposición} % (Pedro: Proposition)


\usepackage[framemethod=default]{mdframed}



\newcommand{\intoo}[2]{\mathopen{]}#1\,;#2\mathclose{[}}
\newcommand{\ud}{\mathop{\mathrm{{}d}}\mathopen{}}
\newcommand{\intff}[2]{\mathopen{[}#1\,;#2\mathclose{]}}
\newtheorem{notation}{Notation}[chapter]


\mdfdefinestyle{exampledefault}{%
rightline=true,innerleftmargin=10,innerrightmargin=10,
frametitlerule=true,frametitlerulecolor=green,
frametitlebackgroundcolor=yellow,
frametitlerulewidth=2pt}


% Theorem box
\newmdenv[skipabove=7pt,
skipbelow=7pt,
backgroundcolor=black!5,
linecolor=ocre,
innerleftmargin=5pt,
innerrightmargin=5pt,
innertopmargin=10pt,%5pt
leftmargin=0cm,
rightmargin=0cm,
innerbottommargin=5pt]{tBox}

% Exercise box	  
\newmdenv[skipabove=7pt,
skipbelow=7pt,
rightline=false,
leftline=true,
topline=false,
bottomline=false,
backgroundcolor=ocre!10,
linecolor=ocre,
innerleftmargin=5pt,
innerrightmargin=5pt,
innertopmargin=10pt,%5pt
innerbottommargin=5pt,
leftmargin=0cm,
rightmargin=0cm,
linewidth=4pt]{eBox}	

% Definition box
\newmdenv[skipabove=7pt,
skipbelow=7pt,
rightline=false,
leftline=true,
topline=false,
bottomline=false,
linecolor=ocre,
innerleftmargin=5pt,
innerrightmargin=5pt,
innertopmargin=10pt,%0pt
leftmargin=0cm,
rightmargin=0cm,
linewidth=4pt,
innerbottommargin=0pt]{dBox}	

% Corollary box
\newmdenv[skipabove=7pt,
skipbelow=7pt,
rightline=false,
leftline=true,
topline=false,
bottomline=false,
linecolor=gray,
backgroundcolor=black!5,
innerleftmargin=5pt,
innerrightmargin=5pt,
innertopmargin=10pt,%5pt
leftmargin=0cm,
rightmargin=0cm,
linewidth=4pt,
innerbottommargin=5pt]{cBox}

% Crea un entorno para cada tipo de theorem y le asigna un estilo 
% con ayuda de las cajas coloreadas anteriores
\newenvironment{theorem}{\begin{tBox}\begin{theoremeT}}{\end{theoremeT}\end{tBox}}
\newenvironment{exercise}{\begin{eBox}\begin{exerciseT}}{\hfill{\color{ocre}\tiny\ensuremath{\blacksquare}}\end{exerciseT}\end{eBox}}				  
\newenvironment{definition}{\begin{dBox}\begin{definitionT}}{\end{definitionT}\end{dBox}}	
\newenvironment{example}{\begin{exampleT}}{\hfill{\tiny\ensuremath{\blacksquare}}\end{exampleT}}		
\newenvironment{corollary}{\begin{cBox}\begin{corollaryT}}{\end{corollaryT}\end{cBox}}	

%	ENVIRONMENT remark
\newenvironment{remark}{\par\vspace{10pt}\small 
% Espacio blanco vertical sobre la nota y tamaño de fuente menor
\begin{list}{}{
\leftmargin=35pt % Indentación sobre la izquierda
\rightmargin=25pt}\item\ignorespaces % Indentación sobre la derecha
\makebox[-2.5pt]{\begin{tikzpicture}[overlay]
\node[draw=ocre!60,line width=1pt,circle,fill=ocre!25,font=\sffamily\bfseries,inner sep=2pt,outer sep=0pt] at (-15pt,0pt){\textcolor{ocre}{N}}; \end{tikzpicture}} % R naranja en un círculo (Pedro)
\advance\baselineskip -1pt}{\end{list}\vskip5pt} 
% Espaciado de línea más estrecho y espacio en blanco después del comentario


\newenvironment{solutionExe}{\par\vspace{10pt}\small 
\begin{list}{}{
\leftmargin=35pt 
\rightmargin=25pt}\item\ignorespaces 
\makebox[-2.5pt]{\begin{tikzpicture}[overlay]
\node[draw=ocre!60,line width=1pt,circle,fill=ocre!25,font=\sffamily\bfseries,inner sep=2pt,outer sep=0pt] at (-15pt,0pt){\textcolor{ocre}{S}}; \end{tikzpicture}} 
\advance\baselineskip -1pt}{\end{list}\vskip5pt} 

\newenvironment{solutionExa}{\par\vspace{10pt}\small 
\begin{list}{}{
\leftmargin=35pt 
\rightmargin=25pt}\item\ignorespaces 
\makebox[-2.5pt]{\begin{tikzpicture}[overlay]
\node[draw=ocre!60,line width=1pt,circle,fill=ocre!55,font=\sffamily\bfseries,inner sep=2pt,outer sep=0pt] at (-15pt,0pt){\textcolor{ocre}{S}}; \end{tikzpicture}} 
\advance\baselineskip -1pt}{\end{list}\vskip5pt} 

\usepackage{tcolorbox}

\usetikzlibrary{trees}

\theoremstyle{ocrenum}
\newtheorem{solutionT}[dummy]{Solución}  % (Pedro: Corollary)
\newenvironment{solution}{\begin{cBox}\begin{solutionT}}{\end{solutionT}\end{cBox}}	


\newcommand{\tcolorboxsolucion}[2]{%
\begin{tcolorbox}[colback=green!5!white,colframe=green!75!black,title=#1] 
 #2
 %\tcblower  % pone una línea discontinua
\end{tcolorbox}
}% final definición comando

\newtcbox{\mybox}[1][green]{on line,
arc=0pt,outer arc=0pt,colback=#1!10!white,colframe=#1!50!black, boxsep=0pt,left=1pt,right=1pt,top=2pt,bottom=2pt, boxrule=0pt,bottomrule=1pt,toprule=1pt}



\mdfdefinestyle{exampledefault}{%
rightline=true,innerleftmargin=10,innerrightmargin=10,
frametitlerule=true,frametitlerulecolor=green,
frametitlebackgroundcolor=yellow,
frametitlerulewidth=2pt}





\newcommand{\betheorem}{\begin{theorem}}
\newcommand{\eetheorem}{\end{theorem}}
\newcommand{\bedefinition}{\begin{definition}}
\newcommand{\eedefinition}{\end{definition}}

\newcommand{\beremark}{\begin{remark}}
\newcommand{\eeremark}{\end{remark}}
\newcommand{\beexercise}{\begin{exercise}}
\newcommand{\eeexercise}{\end{exercise}}
\newcommand{\beexample}{\begin{example}}
\newcommand{\eeexample}{\end{example}}
\newcommand{\becorollary}{\begin{corollary}}
\newcommand{\eecorollary}{\end{corollary}}


\newcommand{\besolutionExe}{\begin{solutionExe}}
\newcommand{\eesolutionExe}{\end{solutionExe}}
\newcommand{\besolutionExa}{\begin{solutionExa}}
\newcommand{\eesolutionExa}{\end{solutionExa}}


%%%%%%%%


% Caja Salida Markdown
\newmdenv[skipabove=7pt,
skipbelow=7pt,
rightline=false,
leftline=true,
topline=false,
bottomline=false,
backgroundcolor=GreenYellow!10,
linecolor=GreenYellow!80,
innerleftmargin=5pt,
innerrightmargin=5pt,
innertopmargin=10pt,%5pt
innerbottommargin=5pt,
leftmargin=0cm,
rightmargin=0cm,
linewidth=4pt]{mBox}	

%% RMarkdown
\newenvironment{markdownsal}{\begin{mBox}}{\end{mBox}}	

\newcommand{\bmarkdownsal}{\begin{markdownsal}}
\newcommand{\emarkdownsal}{\end{markdownsal}}


\usepackage{array}
\usepackage{multirow}
\usepackage{wrapfig}
\usepackage{colortbl}
\usepackage{pdflscape}
\usepackage{tabu}
\usepackage{threeparttable}
\usepackage{subfig} %new
%\usepackage{booktabs,dcolumn,rotating,thumbpdf,longtable}
\usepackage{dcolumn,rotating}  %new
\usepackage[graphicx]{realboxes} %new de: https://stackoverflow.com/questions/51633434/prevent-pagebreak-in-kableextra-landscape-table

%define el interlineado vertical
%\renewcommand{\baselinestretch}{1.5}

%define etiqueta para las Tablas o Cuadros
%\renewcommand\spanishtablename{Tabla}

%%\bibliographystyle{plain} %new no necesario


%%%%%%%%%%%% PARA USO CON biblatex
% \DefineBibliographyStrings{english}{%
%   backrefpage = {ver pag.\adddot},%
%   backrefpages = {ver pags.\adddot}%
% }

% \DefineBibliographyStrings{spanish}{%
%   backrefpage = {ver pag.\adddot},%
%   backrefpages = {ver pags.\adddot}%
% }
% 
% \DeclareFieldFormat{pagerefformat}{\mkbibparens{{\color{red}\mkbibemph{#1}}}}
% \renewbibmacro*{pageref}{%
%   \iflistundef{pageref}
%     {}
%     {\printtext[pagerefformat]{%
%        \ifnumgreater{\value{pageref}}{1}
%          {\bibstring{backrefpages}\ppspace}
%          {\bibstring{backrefpage}\ppspace}%
%        \printlist[pageref][-\value{listtotal}]{pageref}}}}
% 
%%% de kableExtra
\usepackage{booktabs}
\usepackage{longtable}
%\usepackage{array}
%\usepackage{multirow}
%\usepackage{wrapfig}
%\usepackage{float}
%\usepackage{colortbl}
%\usepackage{pdflscape}
%\usepackage{tabu}
%\usepackage{threeparttable}
\usepackage{threeparttablex}
\usepackage[normalem]{ulem}
\usepackage{makecell}
%\usepackage{xcolor}

%%%%%%% fin: latex_preambulo.tex PLLC


\begin{document}

\bibliographystyle{flexbib}



\raggedbottom

\ifdefined\ifprincipal
\else
\setlength{\parindent}{1em}
\pagestyle{fancy}
\setcounter{tocdepth}{4}
\tableofcontents

\fi

\ifdefined\ifdoblecara
\fancyhead{}{}
\fancyhead[LE,RO]{\scriptsize\rightmark}
\fancyfoot[LO,RE]{\scriptsize\slshape \leftmark}
\fancyfoot[C]{}
\fancyfoot[LE,RO]{\footnotesize\thepage}
\else
\fancyhead{}{}
\fancyhead[RO]{\scriptsize\rightmark}
\fancyfoot[LO]{\scriptsize\slshape \leftmark}
\fancyfoot[C]{}
\fancyfoot[RO]{\footnotesize\thepage}
\fi

\renewcommand{\headrulewidth}{0.4pt}
\renewcommand{\footrulewidth}{0.4pt}

\hypertarget{capuxedtulo-1-introducciuxf3n}{%
\chapter{Capítulo 1: Introducción}\label{capuxedtulo-1-introducciuxf3n}}

\hypertarget{introducciuxf3n}{%
\section{Introducción}\label{introducciuxf3n}}

Posible estructura de la introducción obtenida de ``Spatial Prediction
of Wildfire Susceptibility Using Field Survey GPS Data and Machine
Learning Approaches, Omid Ghorbanzadeh, Khalil Valizadeh Kamran, Thomas
Blaschke, Jagannath Aryal, Amin Naboureh, Jamshid Einali and Jinhu
Bian''

Superficie forestal en Andalucía. Principales usos de los bosques
Biodiversidad Andalucía es la segunda comunidad de España con más
terreno forestal, con 4.325.378 ha de suelo forestal que suponen el
49.37\% de su superficie.

(\url{https://www.miteco.gob.es/es/biodiversidad/temas/inventarios-nacionales/inventario-forestal-nacional/superficie_por_uso.html})

En 2022, 15.786,64 hectarias fueron afectadas por el fuego en Andalucía,
practicamente el doble de la media anual de los 10 años anteriores,
8873,68 hectarias {[}Memoria Plan INFOCA 2022{]}.

Riesgo para la biodiversidad vegetal y animal. Daño a la infraestructura
y riesgo para la vida de las personas. Costes incendios forestales
{[}Barómetro de las catastrofes Naturales en España 2022{]}

En el Plan Estratégico Estatal del Patrimonio Natural y de la
Biodiversidad a 2030 {[}Cita{]} ``identifica las principales presiones y
amenazas, entre las cuales destaca el aumento de la superficie forestal
afectada por incendios forestales en 2022''.

Incendios forestales (\textgreater1ha) vs conatos (\textless=1ha).

Incendios forestales: ocurren en monte (parte incendio).

Las causas de los incendios de pueden agregar en dos grupos: causas
naturales o causas antropogénicas. Las últimas se dividen en accidentes,
neglicencias o intencionados. {[}\ldots{} Explicación breve cada una{]}.
En 2022, solo el 2.23\% de los incendios fueron debidos a causas
naturales, mientras que un 35.01\% fue debido a negligencias, un 40.91\%
fue intencionado y un 6.47\% fue debido a causas accidentales.

Estas cifras muestran la importancia del factor humano en el estudio de
los incendios forestales. En {[}Human-caused wildfire risk rating for
prevention planning in Spain,Jesús Martínez, Cristina Vega-Garcia,
Emilio Chuvieco,2008{]} se ha estudiado el factor humano en la causa de
incendios forestales en España, detectando las variables más
significativas en los modelos de predicción construidos (alcanzando una
tasa de acierto del 76\% en los datos test).

En el presente trabajo se pretende estudiar el riesgo de que los
distintos puntos del territorio Andaluz se vean afectados por un
incendio forestal en un momento concreto a través de {[}26{]} variables
explicativas (antropogénicas, hidrográficas, topográficas,
meteorológicas y vegetación). A diferencia de la mayoría de estudios
similares realizados, se aborda el problema desde una perspectiva
dinámica. Esto significa que no se pretende obtener un valor
estacionario o estático del riesgo de incendio forestal en cada punto,
si no que se busca estimar el riesgo de que una zona concreta se vea
afectada por un incendio forestal en un momento concreto. Para ello, se
consideran los valores de las variables correspondiente al momento
concreto de la observación (usando la información más desagregada
temporal y especialmente de la que se ha podido disponer).

Si bien las variables topográficas, hidrográfica y las ligadas a la
infraestructura se han considerado constantes en los 20 años del periodo
en estudio (por considerarse estructurales), las variables ligadas a las
condiciones climáticas, al número de habitantes del municipio y al
estado de la vegetación se consideran relativas al momento concreto de
cada observación.

Población: variaciones en las tendencias demográficas tienen un impacto
importante sobre los regímenes de incendios que se producen {[}Añadir
fuente{]}

Variables climáticas y del estado de la vegetación: Variables con una
gran variabilidad temporal que tienen un impacto directo en la aparición
y propagación de los incendios forestales.

Este enfoque hace posible que se tengan en cuenta las variaciones que se
producen en las variables a lo largo del tiempo (cambios en el número de
habitantes, en las condiciones climáticas y en el estado de la
vegetación), añadiendo así una nueva dimensión al problema.

La po

en la que se consideran los valores de las variables correspondientes al
momento concreto de la observación

Spain could be considered as a key area for wildfire modeling since it
is, by far, the most fire-affected territory within the European Union
-\textgreater{} Alguna característica relevante de Andalucía
¿Biodiversidad?

INFOCA

\hypertarget{objetivos}{%
\section{Objetivos}\label{objetivos}}

El objetivo de esta investigación será construir modelos que permitan
predecir el riesgo de incendio forestal en la Comunidad Autónoma de
Andalucía.

Subobjetivos:

\begin{enumerate}
\def\labelenumi{\arabic{enumi}.}
\item
  Construir un conjunto de datos que permita la realización de análisis
  y la posterior construcción de modelos de Machine Learning para la
  predicción del riesgo de incendio forestal en Andalucía a partir de un
  estudio previo del problema.
\item
  Modelizar el riesgo de de incendio forestal usando distintos
  algoritmos de ML y comparar sus resultados
\item
  Analizar potenciales casos de interés.
\end{enumerate}

\hypertarget{hipuxf3tesis}{%
\section{Hipótesis}\label{hipuxf3tesis}}

``Spatial Prediction of Wildfire Susceptibility Using Field Survey GPS
Data and Machine Learning Approaches, Omid Ghorbanzadeh, Khalil
Valizadeh Kamran, Thomas Blaschke, Jagannath Aryal, Amin Naboureh,
Jamshid Einali and Jinhu Bian''

\begin{figure}
\centering
\includegraphics{D:/usuario/Documents/Universidad/5º/TFG - organizado/memoriaTFE/private/hipotesis_efecto_variables.png}
\caption{Spatial Prediction of Wildfire Susceptibility Using Field
Survey GPS Data and Machine Learning Approaches, Omid Ghorbanzadeh,
Khalil Valizadeh Kamran, Thomas Blaschke, Jagannath Aryal, Amin
Naboureh, Jamshid Einali and Jinhu Bian.}
\end{figure}

\hypertarget{revisiuxf3n-bibliogruxe1fica}{%
\section{Revisión bibliográfica}\label{revisiuxf3n-bibliogruxe1fica}}

``Spain on fire: A novel wildfire risk assessment model based on image
satellite processing and atmospheric information, Helena Liz-López ,
Javier Huertas-Tato a , Jorge Pérez-Aracil, Carlos Casanova-Mateo, Julia
Sanz-Justo, David Camacho''

``A review of machine learning applications in wildfire science and
management Piyush Jain, Sean C.P. Coogan, Sriram Ganapathi Subramanian,
Mark Crowley, Steve Taylor, and Mike D. Flannigan''

``Los incendios forestales en Andalucía: investigación exploratoria y
modelos explicativos'' Oliver Gutiérrez-Hernández (1*), José María
Senciales-González (2), Luis V. García (1)

\hypertarget{anuxe1lisis-del-problema}{%
\section{Análisis del problema}\label{anuxe1lisis-del-problema}}

El área de estudio (Figura 1) abarca el conjunto de la comunidad
autónoma de Andalucía (España), la región más meridional de la península
Ibérica, un territorio de 87.268 km2, donde el 50,8 \% de la superficie
está ocupada por usos y cubiertas forestales.

-\textgreater{} MAPA

2002-2022 Razones: - Disponibilidad datos - Cambios en los regímenes de
incendios

\bibliography{bib/library.bib,bib/paquetes.bib}


%


\end{document}
