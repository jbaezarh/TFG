\documentclass[12pt,a4paper,]{book}
\def\ifdoblecara{} %% set to true
\def\ifprincipal{} %% set to true
\let\ifprincipal\undefined %% set to false
\def\ifcitapandoc{} %% set to true
\let\ifcitapandoc\undefined %% set to false
\usepackage{lmodern}
% sin fontmathfamily
\usepackage{amssymb,amsmath}
\usepackage{ifxetex,ifluatex}
%\usepackage{fixltx2e} % provides \textsubscript %PLLC
\ifnum 0\ifxetex 1\fi\ifluatex 1\fi=0 % if pdftex
  \usepackage[T1]{fontenc}
  \usepackage[utf8]{inputenc}
\else % if luatex or xelatex
  \ifxetex
    \usepackage{mathspec}
  \else
    \usepackage{fontspec}
  \fi
  \defaultfontfeatures{Ligatures=TeX,Scale=MatchLowercase}
\fi
% use upquote if available, for straight quotes in verbatim environments
\IfFileExists{upquote.sty}{\usepackage{upquote}}{}
% use microtype if available
\IfFileExists{microtype.sty}{%
\usepackage{microtype}
\UseMicrotypeSet[protrusion]{basicmath} % disable protrusion for tt fonts
}{}
\usepackage[margin = 2.5cm]{geometry}
\usepackage{hyperref}
\hypersetup{unicode=true,
            pdfauthor={Nombre Completo Autor},
              pdfborder={0 0 0},
              breaklinks=true}
\urlstyle{same}  % don't use monospace font for urls
%
\usepackage[usenames,dvipsnames]{xcolor}  %new PLLC
\IfFileExists{parskip.sty}{%
\usepackage{parskip}
}{% else
\setlength{\parindent}{0pt}
\setlength{\parskip}{6pt plus 2pt minus 1pt}
}
\setlength{\emergencystretch}{3em}  % prevent overfull lines
\providecommand{\tightlist}{%
  \setlength{\itemsep}{0pt}\setlength{\parskip}{0pt}}
\setcounter{secnumdepth}{5}
% Redefines (sub)paragraphs to behave more like sections
\ifx\paragraph\undefined\else
\let\oldparagraph\paragraph
\renewcommand{\paragraph}[1]{\oldparagraph{#1}\mbox{}}
\fi
\ifx\subparagraph\undefined\else
\let\oldsubparagraph\subparagraph
\renewcommand{\subparagraph}[1]{\oldsubparagraph{#1}\mbox{}}
\fi

%%% Use protect on footnotes to avoid problems with footnotes in titles
\let\rmarkdownfootnote\footnote%
\def\footnote{\protect\rmarkdownfootnote}


  \title{}
    \author{Nombre Completo Autor}
      \date{18/11/2021}


%%%%%%% inicio: latex_preambulo.tex PLLC


%% UTILIZA CODIFICACIÓN UTF-8
%% MODIFICARLO CONVENIENTEMENTE PARA USARLO CON OTRAS CODIFICACIONES


%\usepackage[spanish,es-nodecimaldot,es-noshorthands]{babel}
\usepackage[spanish,es-nodecimaldot,es-noshorthands,es-tabla]{babel}
% Ver: es-tabla (en: https://osl.ugr.es/CTAN/macros/latex/contrib/babel-contrib/spanish/spanish.pdf)
% es-tabla (en: https://tex.stackexchange.com/questions/80443/change-the-word-table-in-table-captions)
\usepackage[spanish, plain, datebegin,sortcompress,nocomment,
noabstract]{flexbib}
 
\usepackage{float}
\usepackage{placeins}
\usepackage{fancyhdr}
% Solucion: ! LaTeX Error: Command \counterwithout already defined.
% https://tex.stackexchange.com/questions/425600/latex-error-command-counterwithout-already-defined
\let\counterwithout\relax
\let\counterwithin\relax
\usepackage{chngcntr}
%\usepackage{microtype}  %antes en template PLLC
\usepackage[utf8]{inputenc}
\usepackage[T1]{fontenc} % Usa codificación 8-bit que tiene 256 glyphs

%\usepackage[dvipsnames]{xcolor}
%\usepackage[usenames,dvipsnames]{xcolor}  %new
\usepackage{pdfpages}
%\usepackage{natbib}




% Para portada: latex_paginatitulo_mod_ST02.tex (inicio)
\usepackage{tikz}
\usepackage{epigraph}
\input{portadas/latex_paginatitulo_mod_ST02_add.sty}
% Para portada: latex_paginatitulo_mod_ST02.tex (fin)

% Para portada: latex_paginatitulo_mod_OV01.tex (inicio)
\usepackage{cpimod}
% Para portada: latex_paginatitulo_mod_OV01.tex (fin)

% Para portada: latex_paginatitulo_mod_OV03.tex (inicio)
\usepackage{KTHEEtitlepage}
% Para portada: latex_paginatitulo_mod_OV03.tex (fin)

\renewcommand{\contentsname}{Índice}
\renewcommand{\listfigurename}{Índice de figuras}
\renewcommand{\listtablename}{Índice de tablas}
\newcommand{\bcols}{}
\newcommand{\ecols}{}
\newcommand{\bcol}[1]{\begin{minipage}{#1\linewidth}}
\newcommand{\ecol}{\end{minipage}}
\newcommand{\balertblock}[1]{\begin{alertblock}{#1}}
\newcommand{\ealertblock}{\end{alertblock}}
\newcommand{\bitemize}{\begin{itemize}}
\newcommand{\eitemize}{\end{itemize}}
\newcommand{\benumerate}{\begin{enumerate}}
\newcommand{\eenumerate}{\end{enumerate}}
\newcommand{\saltopagina}{\newpage}
\newcommand{\bcenter}{\begin{center}}
\newcommand{\ecenter}{\end{center}}
\newcommand{\beproof}{\begin{proof}} %new
\newcommand{\eeproof}{\end{proof}} %new
%De: https://texblog.org/2007/11/07/headerfooter-in-latex-with-fancyhdr/
% \fancyhead
% E: Even page
% O: Odd page
% L: Left field
% C: Center field
% R: Right field
% H: Header
% F: Footer
%\fancyhead[CO,CE]{Resultados}

%OPCION 1
% \fancyhead[LE,RO]{\slshape \rightmark}
% \fancyhead[LO,RE]{\slshape \leftmark}
% \fancyfoot[C]{\thepage}
% \renewcommand{\headrulewidth}{0.4pt}
% \renewcommand{\footrulewidth}{0pt}

%OPCION 2
% \fancyhead[LE,RO]{\slshape \rightmark}
% \fancyfoot[LO,RE]{\slshape \leftmark}
% \fancyfoot[LE,RO]{\thepage}
% \renewcommand{\headrulewidth}{0.4pt}
% \renewcommand{\footrulewidth}{0.4pt}
%%%%%%%%%%
\usepackage{calc,amsfonts}
% Elimina la cabecera de páginas impares vacías al finalizar los capítulos
\usepackage{emptypage}
\makeatletter

%\definecolor{ocre}{RGB}{25,25,243} % Define el color azul (naranja) usado para resaltar algunas salidas
\definecolor{ocre}{RGB}{0,0,0} % Define el color a negro (aparece en los teoremas

%\usepackage{calc} 


%era if(csl-refs) con dolares
% metodobib: true


\usepackage{lipsum}

%\usepackage{tikz} % Requerido para dibujar formas personalizadas

%\usepackage{amsmath,amsthm,amssymb,amsfonts}
\usepackage{amsthm}


% Boxed/framed environments
\newtheoremstyle{ocrenumbox}% % Theorem style name
{0pt}% Space above
{0pt}% Space below
{\normalfont}% % Body font
{}% Indent amount
{\small\bf\sffamily\color{ocre}}% % Theorem head font
{\;}% Punctuation after theorem head
{0.25em}% Space after theorem head
{\small\sffamily\color{ocre}\thmname{#1}\nobreakspace\thmnumber{\@ifnotempty{#1}{}\@upn{#2}}% Theorem text (e.g. Theorem 2.1)
\thmnote{\nobreakspace\the\thm@notefont\sffamily\bfseries\color{black}---\nobreakspace#3.}} % Optional theorem note
\renewcommand{\qedsymbol}{$\blacksquare$}% Optional qed square

\newtheoremstyle{blacknumex}% Theorem style name
{5pt}% Space above
{5pt}% Space below
{\normalfont}% Body font
{} % Indent amount
{\small\bf\sffamily}% Theorem head font
{\;}% Punctuation after theorem head
{0.25em}% Space after theorem head
{\small\sffamily{\tiny\ensuremath{\blacksquare}}\nobreakspace\thmname{#1}\nobreakspace\thmnumber{\@ifnotempty{#1}{}\@upn{#2}}% Theorem text (e.g. Theorem 2.1)
\thmnote{\nobreakspace\the\thm@notefont\sffamily\bfseries---\nobreakspace#3.}}% Optional theorem note

\newtheoremstyle{blacknumbox} % Theorem style name
{0pt}% Space above
{0pt}% Space below
{\normalfont}% Body font
{}% Indent amount
{\small\bf\sffamily}% Theorem head font
{\;}% Punctuation after theorem head
{0.25em}% Space after theorem head
{\small\sffamily\thmname{#1}\nobreakspace\thmnumber{\@ifnotempty{#1}{}\@upn{#2}}% Theorem text (e.g. Theorem 2.1)
\thmnote{\nobreakspace\the\thm@notefont\sffamily\bfseries---\nobreakspace#3.}}% Optional theorem note

% Non-boxed/non-framed environments
\newtheoremstyle{ocrenum}% % Theorem style name
{5pt}% Space above
{5pt}% Space below
{\normalfont}% % Body font
{}% Indent amount
{\small\bf\sffamily\color{ocre}}% % Theorem head font
{\;}% Punctuation after theorem head
{0.25em}% Space after theorem head
{\small\sffamily\color{ocre}\thmname{#1}\nobreakspace\thmnumber{\@ifnotempty{#1}{}\@upn{#2}}% Theorem text (e.g. Theorem 2.1)
\thmnote{\nobreakspace\the\thm@notefont\sffamily\bfseries\color{black}---\nobreakspace#3.}} % Optional theorem note
\renewcommand{\qedsymbol}{$\blacksquare$}% Optional qed square
\makeatother



% Define el estilo texto theorem para cada tipo definido anteriormente
\newcounter{dummy} 
\numberwithin{dummy}{section}
\theoremstyle{ocrenumbox}
\newtheorem{theoremeT}[dummy]{Teorema}  % (Pedro: Theorem)
\newtheorem{problem}{Problema}[chapter]  % (Pedro: Problem)
\newtheorem{exerciseT}{Ejercicio}[chapter] % (Pedro: Exercise)
\theoremstyle{blacknumex}
\newtheorem{exampleT}{Ejemplo}[chapter] % (Pedro: Example)
\theoremstyle{blacknumbox}
\newtheorem{vocabulary}{Vocabulario}[chapter]  % (Pedro: Vocabulary)
\newtheorem{definitionT}{Definición}[section]  % (Pedro: Definition)
\newtheorem{corollaryT}[dummy]{Corolario}  % (Pedro: Corollary)
\theoremstyle{ocrenum}
\newtheorem{proposition}[dummy]{Proposición} % (Pedro: Proposition)


\usepackage[framemethod=default]{mdframed}



\newcommand{\intoo}[2]{\mathopen{]}#1\,;#2\mathclose{[}}
\newcommand{\ud}{\mathop{\mathrm{{}d}}\mathopen{}}
\newcommand{\intff}[2]{\mathopen{[}#1\,;#2\mathclose{]}}
\newtheorem{notation}{Notation}[chapter]


\mdfdefinestyle{exampledefault}{%
rightline=true,innerleftmargin=10,innerrightmargin=10,
frametitlerule=true,frametitlerulecolor=green,
frametitlebackgroundcolor=yellow,
frametitlerulewidth=2pt}


% Theorem box
\newmdenv[skipabove=7pt,
skipbelow=7pt,
backgroundcolor=black!5,
linecolor=ocre,
innerleftmargin=5pt,
innerrightmargin=5pt,
innertopmargin=10pt,%5pt
leftmargin=0cm,
rightmargin=0cm,
innerbottommargin=5pt]{tBox}

% Exercise box	  
\newmdenv[skipabove=7pt,
skipbelow=7pt,
rightline=false,
leftline=true,
topline=false,
bottomline=false,
backgroundcolor=ocre!10,
linecolor=ocre,
innerleftmargin=5pt,
innerrightmargin=5pt,
innertopmargin=10pt,%5pt
innerbottommargin=5pt,
leftmargin=0cm,
rightmargin=0cm,
linewidth=4pt]{eBox}	

% Definition box
\newmdenv[skipabove=7pt,
skipbelow=7pt,
rightline=false,
leftline=true,
topline=false,
bottomline=false,
linecolor=ocre,
innerleftmargin=5pt,
innerrightmargin=5pt,
innertopmargin=10pt,%0pt
leftmargin=0cm,
rightmargin=0cm,
linewidth=4pt,
innerbottommargin=0pt]{dBox}	

% Corollary box
\newmdenv[skipabove=7pt,
skipbelow=7pt,
rightline=false,
leftline=true,
topline=false,
bottomline=false,
linecolor=gray,
backgroundcolor=black!5,
innerleftmargin=5pt,
innerrightmargin=5pt,
innertopmargin=10pt,%5pt
leftmargin=0cm,
rightmargin=0cm,
linewidth=4pt,
innerbottommargin=5pt]{cBox}

% Crea un entorno para cada tipo de theorem y le asigna un estilo 
% con ayuda de las cajas coloreadas anteriores
\newenvironment{theorem}{\begin{tBox}\begin{theoremeT}}{\end{theoremeT}\end{tBox}}
\newenvironment{exercise}{\begin{eBox}\begin{exerciseT}}{\hfill{\color{ocre}\tiny\ensuremath{\blacksquare}}\end{exerciseT}\end{eBox}}				  
\newenvironment{definition}{\begin{dBox}\begin{definitionT}}{\end{definitionT}\end{dBox}}	
\newenvironment{example}{\begin{exampleT}}{\hfill{\tiny\ensuremath{\blacksquare}}\end{exampleT}}		
\newenvironment{corollary}{\begin{cBox}\begin{corollaryT}}{\end{corollaryT}\end{cBox}}	

%	ENVIRONMENT remark
\newenvironment{remark}{\par\vspace{10pt}\small 
% Espacio blanco vertical sobre la nota y tamaño de fuente menor
\begin{list}{}{
\leftmargin=35pt % Indentación sobre la izquierda
\rightmargin=25pt}\item\ignorespaces % Indentación sobre la derecha
\makebox[-2.5pt]{\begin{tikzpicture}[overlay]
\node[draw=ocre!60,line width=1pt,circle,fill=ocre!25,font=\sffamily\bfseries,inner sep=2pt,outer sep=0pt] at (-15pt,0pt){\textcolor{ocre}{N}}; \end{tikzpicture}} % R naranja en un círculo (Pedro)
\advance\baselineskip -1pt}{\end{list}\vskip5pt} 
% Espaciado de línea más estrecho y espacio en blanco después del comentario


\newenvironment{solutionExe}{\par\vspace{10pt}\small 
\begin{list}{}{
\leftmargin=35pt 
\rightmargin=25pt}\item\ignorespaces 
\makebox[-2.5pt]{\begin{tikzpicture}[overlay]
\node[draw=ocre!60,line width=1pt,circle,fill=ocre!25,font=\sffamily\bfseries,inner sep=2pt,outer sep=0pt] at (-15pt,0pt){\textcolor{ocre}{S}}; \end{tikzpicture}} 
\advance\baselineskip -1pt}{\end{list}\vskip5pt} 

\newenvironment{solutionExa}{\par\vspace{10pt}\small 
\begin{list}{}{
\leftmargin=35pt 
\rightmargin=25pt}\item\ignorespaces 
\makebox[-2.5pt]{\begin{tikzpicture}[overlay]
\node[draw=ocre!60,line width=1pt,circle,fill=ocre!55,font=\sffamily\bfseries,inner sep=2pt,outer sep=0pt] at (-15pt,0pt){\textcolor{ocre}{S}}; \end{tikzpicture}} 
\advance\baselineskip -1pt}{\end{list}\vskip5pt} 

\usepackage{tcolorbox}

\usetikzlibrary{trees}

\theoremstyle{ocrenum}
\newtheorem{solutionT}[dummy]{Solución}  % (Pedro: Corollary)
\newenvironment{solution}{\begin{cBox}\begin{solutionT}}{\end{solutionT}\end{cBox}}	


\newcommand{\tcolorboxsolucion}[2]{%
\begin{tcolorbox}[colback=green!5!white,colframe=green!75!black,title=#1] 
 #2
 %\tcblower  % pone una línea discontinua
\end{tcolorbox}
}% final definición comando

\newtcbox{\mybox}[1][green]{on line,
arc=0pt,outer arc=0pt,colback=#1!10!white,colframe=#1!50!black, boxsep=0pt,left=1pt,right=1pt,top=2pt,bottom=2pt, boxrule=0pt,bottomrule=1pt,toprule=1pt}



\mdfdefinestyle{exampledefault}{%
rightline=true,innerleftmargin=10,innerrightmargin=10,
frametitlerule=true,frametitlerulecolor=green,
frametitlebackgroundcolor=yellow,
frametitlerulewidth=2pt}





\newcommand{\betheorem}{\begin{theorem}}
\newcommand{\eetheorem}{\end{theorem}}
\newcommand{\bedefinition}{\begin{definition}}
\newcommand{\eedefinition}{\end{definition}}

\newcommand{\beremark}{\begin{remark}}
\newcommand{\eeremark}{\end{remark}}
\newcommand{\beexercise}{\begin{exercise}}
\newcommand{\eeexercise}{\end{exercise}}
\newcommand{\beexample}{\begin{example}}
\newcommand{\eeexample}{\end{example}}
\newcommand{\becorollary}{\begin{corollary}}
\newcommand{\eecorollary}{\end{corollary}}


\newcommand{\besolutionExe}{\begin{solutionExe}}
\newcommand{\eesolutionExe}{\end{solutionExe}}
\newcommand{\besolutionExa}{\begin{solutionExa}}
\newcommand{\eesolutionExa}{\end{solutionExa}}


%%%%%%%%


% Caja Salida Markdown
\newmdenv[skipabove=7pt,
skipbelow=7pt,
rightline=false,
leftline=true,
topline=false,
bottomline=false,
backgroundcolor=GreenYellow!10,
linecolor=GreenYellow!80,
innerleftmargin=5pt,
innerrightmargin=5pt,
innertopmargin=10pt,%5pt
innerbottommargin=5pt,
leftmargin=0cm,
rightmargin=0cm,
linewidth=4pt]{mBox}	

%% RMarkdown
\newenvironment{markdownsal}{\begin{mBox}}{\end{mBox}}	

\newcommand{\bmarkdownsal}{\begin{markdownsal}}
\newcommand{\emarkdownsal}{\end{markdownsal}}


\usepackage{array}
\usepackage{multirow}
\usepackage{wrapfig}
\usepackage{colortbl}
\usepackage{pdflscape}
\usepackage{tabu}
\usepackage{threeparttable}
\usepackage{subfig} %new
%\usepackage{booktabs,dcolumn,rotating,thumbpdf,longtable}
\usepackage{dcolumn,rotating}  %new
\usepackage[graphicx]{realboxes} %new de: https://stackoverflow.com/questions/51633434/prevent-pagebreak-in-kableextra-landscape-table

%define el interlineado vertical
%\renewcommand{\baselinestretch}{1.5}

%define etiqueta para las Tablas o Cuadros
%\renewcommand\spanishtablename{Tabla}

%%\bibliographystyle{plain} %new no necesario


%%%%%%%%%%%% PARA USO CON biblatex
% \DefineBibliographyStrings{english}{%
%   backrefpage = {ver pag.\adddot},%
%   backrefpages = {ver pags.\adddot}%
% }

% \DefineBibliographyStrings{spanish}{%
%   backrefpage = {ver pag.\adddot},%
%   backrefpages = {ver pags.\adddot}%
% }
% 
% \DeclareFieldFormat{pagerefformat}{\mkbibparens{{\color{red}\mkbibemph{#1}}}}
% \renewbibmacro*{pageref}{%
%   \iflistundef{pageref}
%     {}
%     {\printtext[pagerefformat]{%
%        \ifnumgreater{\value{pageref}}{1}
%          {\bibstring{backrefpages}\ppspace}
%          {\bibstring{backrefpage}\ppspace}%
%        \printlist[pageref][-\value{listtotal}]{pageref}}}}
% 
%%% de kableExtra
\usepackage{booktabs}
\usepackage{longtable}
%\usepackage{array}
%\usepackage{multirow}
%\usepackage{wrapfig}
%\usepackage{float}
%\usepackage{colortbl}
%\usepackage{pdflscape}
%\usepackage{tabu}
%\usepackage{threeparttable}
\usepackage{threeparttablex}
\usepackage[normalem]{ulem}
\usepackage{makecell}
%\usepackage{xcolor}

%%%%%%% fin: latex_preambulo.tex PLLC


\begin{document}

\bibliographystyle{flexbib}



\raggedbottom

\ifdefined\ifprincipal
\else
\setlength{\parindent}{1em}
\pagestyle{fancy}
\setcounter{tocdepth}{4}
\tableofcontents

\fi

\ifdefined\ifdoblecara
\fancyhead{}{}
\fancyhead[LE,RO]{\scriptsize\rightmark}
\fancyfoot[LO,RE]{\scriptsize\slshape \leftmark}
\fancyfoot[C]{}
\fancyfoot[LE,RO]{\footnotesize\thepage}
\else
\fancyhead{}{}
\fancyhead[RO]{\scriptsize\rightmark}
\fancyfoot[LO]{\scriptsize\slshape \leftmark}
\fancyfoot[C]{}
\fancyfoot[RO]{\footnotesize\thepage}
\fi

\renewcommand{\headrulewidth}{0.4pt}
\renewcommand{\footrulewidth}{0.4pt}

\hypertarget{introducciuxf3n}{%
\chapter{Introducción}\label{introducciuxf3n}}

El fuego es un factor natural clave en los ecosistemas
terrestres.\footnote{El contenido de los cuatro primeros párrafos de
  esta introducción está extraído principalmente de \citet{MataixCerda}}
Tan solo necesita de tres elementos básico que se encuentran en
abundancia en la superficie de la Tierra: oxígeno, combustible y calor,
por lo que no sorprende que su presencia en este planeta se remonte muy
atrás en el tiempo. Hay pruebas de la existencia de fuego hace 400
millones de años y desde hace 350 millones de años se producen incendios
en la Tierra de forma frecuente. Para tener con qué comparar esta cifra,
las hipótesis más aceptadas indican que los primeros Homo Sapiens datan
de hace unos 200 mil años (FUENTE)

El fuego ha condicionado la evolución y la dispersión de plantas, el
desarrollo de los biomas, la formación de suelos y los ciclos
hidrológicos y erosivos. Se trata, por tanto, de uno de los procesos
platenarios clave. La presencia del fuego en los ecosistemas terrestres
ha dado lugar a numerosas adaptaciones en los seres vivos. En el caso de
los ecosistemas mediterráneos, donde los incendios son frecuentes, la
vegetación ha desarrollado adaptaciones que le permiten adaptarse al
fuego. Un buen ejemplo de esta adaptación al fuego lo encontramos en el
pino carrasco (\emph{Pinus halepensis}), cuyas piñas que cuelgan de las
ramas de la copa solo se abren al calor de las llamas.

El control del fuego fue clave para el desarrollo de la humanidad.
Cocinar alimentos, abrir zonas de cultivos, facilitar el traslado de la
población, quemar restos de cosechas o eliminar plagas son solo algunos
de los usos que se le ha dado al fuego desde que los primeros homínidos
lo descubrieron hace 1,5 millones de años. Sin embargo, la relación del
hombre con el fuego parece haber cambiado radicalmente en los últimos
años.

La industrialización de las sociedades modernas, la sustitución del uso
de biomasa por combustibles fósiles y el éxodo rural junto con el
abandono de la agricultura han provocado el cese de la explotación de
los montes, generando grandes acumulaciones de biomasa que actúa como
combustible para los incendios forestales. De esta forma, el fuego ha
pasado de ser una herramienta para el hombre, como ha sido a lo largo de
la historia, a convertirse en un gran problema ambiental.

A todo esto hay que sumar los efectos del cambio climático, que a través
del aumento de las temperaturas y la disminución de las precipitaciones,
ha provocado que se alargue la estación de incendios y ha aumentado las
situaciones de riesgo alto \citep{incendioCamCLim}.

En España, los incendios forestales figuran en el Plan Estratégico
Estatal del Patrimonio Natural y de la Biodiversidad a 2030 como una de
las principales presiones y amenazas para el patrimonio natural y la
biodiversidad en España, considerándose ``el principal elemento de
degradación de los ecosistemas forestales, con importantes repercusiones
sobre bienes e incluso vidas humanas''.

En Andalucía este problema toma una dimensión especial al tratarse de la
segunda comunidad de España con más terreno forestal (cuenta con
4.325.378 ha de suelo forestal que suponen el 49.37\% de su superficie)
y contar además con el \emph{hotspot} de biodiversidad de Sierra Nevada,
uno de los enclaves con mayor diversidad del continente.

En 2022, 15.786,64 ha fueron afectadas por el fuego en Andalucía,
prácticamente el doble de la media anual de los 10 años anteriores,
8873,68 ha Ese mismo año, el 91.91\% de las actuaciones
forestales\footnote{Según se indica en la Estadística General de
  Incendios Forestales (EGIF), se usa el término actuaciones forestales
  para englobar a conatos (de extensión inferior a 1 ha) e incendios
  forestales (de extensión superior a 1 ha)} con causa es conocida
fueron de origen antrópico. De estos, el 48.20\% fueron debidos a
negligencias, el 38.40\% fueron intencionados y el 13.40\% fueron
accidentales. Tan solo el 5.25\% de las actuaciones forestales que
ocurrieron en Andalucía en 2022 fueron debida a causas naturales
\citep{INFOCA2022}. Estas cifras ponen de manifiesto la importancia de
considerar el factor humano en el estudio de los incendios forestales.

En la siguiente cita, extraída de \citet{incendioCamCLim}, los autores
enfatizan la necesidad de considerar el factor humano en la predicción
de incendios forestales, e indican la existencia de factores que
influyen en que unas determinadas zonas tengan un mayor riesgo de verse
afectas por incendios forestales que otras, entre los que mencionan el
tipo y la configuración de la vegetación:

~~~~~~~~~~\emph{A pesar de la importancia de la meteorología en los
incendios, la capacidad }\\
\hspace*{0.333em}\hspace*{0.333em}\hspace*{0.333em}\hspace*{0.333em}\hspace*{0.333em}\hspace*{0.333em}\emph{predictiva
de la ocurrencia de incendios {[}\ldots{]} en base a variables
meteorológicas}\\
\hspace*{0.333em}\hspace*{0.333em}\hspace*{0.333em}\hspace*{0.333em}\hspace*{0.333em}\hspace*{0.333em}\emph{{[}\ldots{]}
suele ser baja. {[}\ldots{]} Esto es debido a que la mayor parte de los
}\\
\hspace*{0.333em}\hspace*{0.333em}\hspace*{0.333em}\hspace*{0.333em}\hspace*{0.333em}\hspace*{0.333em}\emph{incendios
en España es de de origen humano, lo que dificulta su
predictibilidad.}\\
\hspace*{0.333em}\hspace*{0.333em}\hspace*{0.333em}\hspace*{0.333em}\hspace*{0.333em}\hspace*{0.333em}\emph{Así,
las igniciones no ocurren al azar, ni en el espacio ni en el tiempo.
{[}\ldots{]} }\\
\hspace*{0.333em}\hspace*{0.333em}\hspace*{0.333em}\hspace*{0.333em}\hspace*{0.333em}\hspace*{0.333em}\emph{El
territorio no se quema de manera aleatoria, siendo normal que unas zonas
ardan}\\
\hspace*{0.333em}\hspace*{0.333em}\hspace*{0.333em}\hspace*{0.333em}\hspace*{0.333em}\hspace*{0.333em}\emph{más
que otras. {[}\ldots{]} Unos tipos de vegetación suelen arder más
frecuentemente que }\\
\hspace*{0.333em}\hspace*{0.333em}\hspace*{0.333em}\hspace*{0.333em}\hspace*{0.333em}\hspace*{0.333em}\emph{otros.
{[}\ldots{]} La probabilidad de que un incendio se propague se ve
favorecida por }\\
\hspace*{0.333em}\hspace*{0.333em}\hspace*{0.333em}\hspace*{0.333em}\hspace*{0.333em}\hspace*{0.333em}\emph{la
configuración espacial de las manchas de vegetación que conforman el
paisaje.}

Los incendios forestales son un proceso sumamente complejo en el que se
interrelacionan una gran cantidad de factores como la fuente de
ignición, la composición del combustible, las condiciones meteorológicas
o la orografía del terreno, además de la ya mencionada acción humana.
Desde el estudio de los procesos de combustión que ocurren a escala
molecular al estudio de la propagación de los incendios forestales, el
estudio de los mismos puede abordarse desde numerosos puntos de vista y
con distintos enfoques que van desde la ecología a la física, pasando
por la estadística. Pero cuando se trata de construir los modelos a gran
escala necesarios para llevar a cabo la gestión de los incendios
forestales, las limitaciones computacionales, la cantidad y calidad de
los datos requeridos y la interacción de una enorme cantidad de factores
hacen que la modelización físico-matemática no sea, en muchos, casos un
enfoque factible. Es por ello, que los modelos empíricos y estadísticos
han tomado cada vez más peso en el estudio de los incendios forestales,
aunque su utilidad depende en muchos casos de la calidad y cantidad de
los datos disponibles, así como de la capacidad los modelos de
representar relaciones no lineales presentes en los datos
(\citet{ReviewMLApplicationsWF}).

Understanding and better predicting wildfires is therefore crucial in
several important areas of wildfire management, including emergency
response, ecosystem management, land-use planning, and climate
adaptation to name a few.

El auge experimentado por la inteligencia artificial en las últimas
décadas como consecuencia del desarrollo

El auge de la inteligencia artificial de las últimas décadas promovido
por el desarrollo tecnológico Por ello, no es de extrañar que desde la
década de los 90 se exploren formas de construir mo

hotspot de biodiversidad

A corto plazo, la influencia del fuego en los ecosistemas terrestres es
la eliminación y modificación de la cubirta forestal, que sirve de
combustible a las llamas. Sin embargo, a largo plazo, su influencia

Los incendios forestales son un factor natural de los ecosistemas
terrestres.

El fuego ha dado forma a los ecosistema terrestre, acelerando los
procesos de erosión, acelerando las dinámicas geomorfológicas y, en
definitiva, esculpiendo las formas terrestres

destrucción de vastas áreas de bosque, la degradación del suelo pérdida
de biodiversidad s

, además de poner en riesgo infraestructuras y vidas humanas.

Andalucía es la segunda comunidad de España con más terreno forestal,
con 4.325.378 ha de suelo forestal que suponen el 49.37\% de su
superficie.

En 2022, 15.786,64 hectarias fueron afectadas por el fuego en Andalucía,
practicamente el doble de la media anual de los 10 años anteriores,
8873,68 hectarias {[}Memoria Plan INFOCA 2022{]}.

En el Plan Estratégico Estatal del Patrimonio Natural y de la
Biodiversidad a 2030 {[}Cita{]} ``identifica las principales presiones y
amenazas, entre las cuales destaca el aumento de la superficie forestal
afectada por incendios forestales en 2022''.

\hypertarget{objetivos}{%
\section{Objetivos}\label{objetivos}}

El objetivo de esta investigación es construir modelos de \emph{Machine
Learning} que permitan predecir incendios forestales en la Comunidad
Autónoma de Andalucía. Para abordar dicho objetivo, se realizarán las
siguientes tareas:

\begin{enumerate}
\def\labelenumi{\arabic{enumi}.}
\item
  Construir un conjunto de datos que permita la realización de análisis
  estadísticos y la posterior construcción de modelos de \emph{Machine
  Learning} (ML) para la predicción de incendios forestales en Andalucía
  a partir de un estudio previo del problema.
\item
  Modelizar el riesgo de incendio forestal usando distintos algoritmos
  de ML y comparar sus resultados
\item
  Analizar el desempeño de los modelos en la realidad estudiando
  potenciales casos de interés.
\end{enumerate}

\hypertarget{hipuxf3tesis}{%
\section{Hipótesis}\label{hipuxf3tesis}}

\hypertarget{revisiuxf3n-bibliogruxe1fica}{%
\section{Revisión bibliográfica}\label{revisiuxf3n-bibliogruxe1fica}}

\citet{CortezMorais}: En esta investigación se evalúa el rendimiento de
distintos modelos de Machine Learning, como Máquinas de Soporte
Vectorial, Árboles de Decisión, Regresión Lineal Múltiple, Naive Bayes,
Bosques Aleatorios y Redes Neuronales, para predecir el área quemada por
los incendios forestales en el Parque Natural de Montesinho, en el norte
de Portugal, a partir de información meteorológica. Los mejores
resultados se obtienen para el modelo de SVM considerando 4 variables
explicativas: temperatura, humedad relativa, precipitaciones y viento.

\citet{ReviewMLApplicationsWF}: El artículo revisa las aplicaciones del
Machine Learning (ML) en la ciencia y gestión de incendios forestales,
destacando su uso en seis dominios clave: caracterización de
combustibles, detección y mapeo de incendios; clima y cambio climático;
ocurrencia, susceptibilidad y riesgo de incendios; predicción del
comportamiento del fuego; efectos del fuego; y gestión de incendios. Se
identifican 300 publicaciones hasta finales de 2019, mostrando el uso
frecuente de algoritmos de ML como random forests, MaxEnt, redes
neuronales artificiales, árboles de decisión, máquinas de vectores de
soporte y algoritmos genéticos. La revisión enfatiza las ventajas y
limitaciones de estos métodos y subraya la necesidad de combinar la
experiencia en la ciencia del fuego con técnicas avanzadas de ML para
poder construir modelos realistas y útiles.

\citet{logreg_hcwf}: Se aplica regresión logística en una cuadrícula de
1 × 1 km en la Comunidad Autónoma de Madrid, utilizando datos
socioeconómicos como variables predictoras para representar los factores
antropogénicos relacionados con el riesgo de incendio. También se evalúa
otro enfoque basado en la predicción de la densidad de puntos de
ignición en una cuadrícula de 10 × 10 km, utilizando funciones Kernel.
La ocurrencia histórica de incendios de 2000 a 2005 se utiliza como
variable de respuesta. El rendimiento de los modelos se evalúa con los
incendios ocurridos en 2006 y 2007, obteniendo un AUC de 0.70 y 0.67
para ambos modelos, respectivamente.

\citet{SpainOnFire}: En este artículo se presenta el nuevo Modelo de
Evaluación de Incendios Forestales (WAM, por sus siglas en inglés) que
utiliza deep learning para anticipar el impacto de los incendios a
partir de información meteorológica satelital y el NDVI en Castilla y
León y Andalucía. Las variables respuesta del modelo son el área
quemada, el tiempo de control y extinción, y la cantidad de recursos
humanos, aéreos y pesados necesarios para la extinción del incendio.
Emplea una red convolucional residual que realiza regresiones sobre
variables atmosféricas e índice de verdor. El WAM se preentrena con
100,000 ejemplos de datos sin etiquetar y se ajusta con un pequeño
conjunto de 445 muestras etiquetadas.

\citet{incendiosAndalucia} En este trabajo analiza la superficie
calcinada por los incendios forestales ocurridos entre 1975 y 2013 en la
Comunidad Autónoma de Andalucía en función de 15 variables explicativas:
altitud, insolación, pendiente, precipitación invernal, precipitación
estival, temperatura estival, velocidad del viento, frecuencia del
viento, superficie protegida, superficie de monte público, superficie de
usos forestales, distancia a viario, distancia a zonas pobladas, saldo
demográfico y saldo ganadero. Se aplica un modelo de Regresión Lineal
Múltiple y un Modelo de Regresión Geográficamente Ponderada. Los mejores
resultados se obtienen en el segundo modelo, puesto que permite
considerar la estructura de correlaciones espaciales presentes en los
datos.

\citet{HumanCauseWildFiresSpain} En este artículo se identifican los
factores humanas asociados con un mayor riesgo de incendio forestal en
España y se analiza la distribución espacial de la aparición de
incendios forestales en el país tomando como unidad de estudio el
municipio. Se consideran 108 variables, de las cuales se seleccionan 29
tras un primer estudio exploratorio, las cuales se usan para entrenar un
modelo de regresión logística para estimar la probabilidad de una alta o
baja ocurrencia de incendios. Finalmente, tan solo resultan
significativas 13 de las variables consideradas.

\citet{stojanova2012estimating} \citet{SAYAD2019130}

\bibliography{bib/library.bib,bib/paquetes.bib}


%


\end{document}
