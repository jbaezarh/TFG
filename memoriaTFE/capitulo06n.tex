\documentclass[12pt,a4paper,]{book}
\def\ifdoblecara{} %% set to true
\def\ifprincipal{} %% set to true
\let\ifprincipal\undefined %% set to false
\def\ifcitapandoc{} %% set to true
\let\ifcitapandoc\undefined %% set to false
\usepackage{lmodern}
% sin fontmathfamily
\usepackage{amssymb,amsmath}
\usepackage{ifxetex,ifluatex}
%\usepackage{fixltx2e} % provides \textsubscript %PLLC
\ifnum 0\ifxetex 1\fi\ifluatex 1\fi=0 % if pdftex
  \usepackage[T1]{fontenc}
  \usepackage[utf8]{inputenc}
\else % if luatex or xelatex
  \ifxetex
    \usepackage{mathspec}
  \else
    \usepackage{fontspec}
  \fi
  \defaultfontfeatures{Ligatures=TeX,Scale=MatchLowercase}
\fi
% use upquote if available, for straight quotes in verbatim environments
\IfFileExists{upquote.sty}{\usepackage{upquote}}{}
% use microtype if available
\IfFileExists{microtype.sty}{%
\usepackage{microtype}
\UseMicrotypeSet[protrusion]{basicmath} % disable protrusion for tt fonts
}{}
\usepackage[margin = 2.5cm]{geometry}
\usepackage{hyperref}
\hypersetup{unicode=true,
            pdfauthor={Nombre Completo Autor},
              pdfborder={0 0 0},
              breaklinks=true}
\urlstyle{same}  % don't use monospace font for urls
%
\usepackage[usenames,dvipsnames]{xcolor}  %new PLLC
\IfFileExists{parskip.sty}{%
\usepackage{parskip}
}{% else
\setlength{\parindent}{0pt}
\setlength{\parskip}{6pt plus 2pt minus 1pt}
}
\setlength{\emergencystretch}{3em}  % prevent overfull lines
\providecommand{\tightlist}{%
  \setlength{\itemsep}{0pt}\setlength{\parskip}{0pt}}
\setcounter{secnumdepth}{5}
% Redefines (sub)paragraphs to behave more like sections
\ifx\paragraph\undefined\else
\let\oldparagraph\paragraph
\renewcommand{\paragraph}[1]{\oldparagraph{#1}\mbox{}}
\fi
\ifx\subparagraph\undefined\else
\let\oldsubparagraph\subparagraph
\renewcommand{\subparagraph}[1]{\oldsubparagraph{#1}\mbox{}}
\fi

%%% Use protect on footnotes to avoid problems with footnotes in titles
\let\rmarkdownfootnote\footnote%
\def\footnote{\protect\rmarkdownfootnote}


  \title{}
    \author{Nombre Completo Autor}
      \date{18/11/2021}


%%%%%%% inicio: latex_preambulo.tex PLLC


%% UTILIZA CODIFICACIÓN UTF-8
%% MODIFICARLO CONVENIENTEMENTE PARA USARLO CON OTRAS CODIFICACIONES


%\usepackage[spanish,es-nodecimaldot,es-noshorthands]{babel}
\usepackage[spanish,es-nodecimaldot,es-noshorthands,es-tabla]{babel}
% Ver: es-tabla (en: https://osl.ugr.es/CTAN/macros/latex/contrib/babel-contrib/spanish/spanish.pdf)
% es-tabla (en: https://tex.stackexchange.com/questions/80443/change-the-word-table-in-table-captions)
\usepackage[spanish, plain, datebegin,sortcompress,nocomment,
noabstract]{flexbib}
 
\usepackage{float}
\usepackage{placeins}
\usepackage{fancyhdr}
% Solucion: ! LaTeX Error: Command \counterwithout already defined.
% https://tex.stackexchange.com/questions/425600/latex-error-command-counterwithout-already-defined
\let\counterwithout\relax
\let\counterwithin\relax
\usepackage{chngcntr}
%\usepackage{microtype}  %antes en template PLLC
\usepackage[utf8]{inputenc}
\usepackage[T1]{fontenc} % Usa codificación 8-bit que tiene 256 glyphs

%\usepackage[dvipsnames]{xcolor}
%\usepackage[usenames,dvipsnames]{xcolor}  %new
\usepackage{pdfpages}
%\usepackage{natbib}




% Para portada: latex_paginatitulo_mod_ST02.tex (inicio)
\usepackage{tikz}
\usepackage{epigraph}
\renewcommand\epigraphflush{flushright}
\renewcommand\epigraphsize{\normalsize}
\setlength\epigraphwidth{0.7\textwidth}

\definecolor{titlepagecolor}{cmyk}{1,.60,0,.40}

%\DeclareFixedFont{\titlefont}{T1}{ppl}{b}{it}{0.5in}

% \makeatletter
% \def\printauthor{%
%     {\large \@author}}
% \makeatother
% \author{%
%     Author 1 name \\
%     Department name \\
%     \texttt{email1@example.com}\vspace{20pt} \\
%     Author 2 name \\
%     Department name \\
%     \texttt{email2@example.com}
%     }

% The following code is borrowed from: https://tex.stackexchange.com/a/86310/10898

\newcommand\titlepagedecoration{%
\begin{tikzpicture}[remember picture,overlay,shorten >= -10pt]

\coordinate (aux1) at ([yshift=-15pt]current page.north east);
\coordinate (aux2) at ([yshift=-410pt]current page.north east);
\coordinate (aux3) at ([xshift=-4.5cm]current page.north east);
\coordinate (aux4) at ([yshift=-150pt]current page.north east);

\begin{scope}[titlepagecolor!40,line width=12pt,rounded corners=12pt]
\draw
  (aux1) -- coordinate (a)
  ++(225:5) --
  ++(-45:5.1) coordinate (b);
\draw[shorten <= -10pt]
  (aux3) --
  (a) --
  (aux1);
\draw[opacity=0.6,titlepagecolor,shorten <= -10pt]
  (b) --
  ++(225:2.2) --
  ++(-45:2.2);
\end{scope}
\draw[titlepagecolor,line width=8pt,rounded corners=8pt,shorten <= -10pt]
  (aux4) --
  ++(225:0.8) --
  ++(-45:0.8);
\begin{scope}[titlepagecolor!70,line width=6pt,rounded corners=8pt]
\draw[shorten <= -10pt]
  (aux2) --
  ++(225:3) coordinate[pos=0.45] (c) --
  ++(-45:3.1);
\draw
  (aux2) --
  (c) --
  ++(135:2.5) --
  ++(45:2.5) --
  ++(-45:2.5) coordinate[pos=0.3] (d);   
\draw 
  (d) -- +(45:1);
\end{scope}
\end{tikzpicture}%
}

% Para portada: latex_paginatitulo_mod_ST02.tex (fin)

% Para portada: latex_paginatitulo_mod_OV01.tex (inicio)
\usepackage{cpimod}
% Para portada: latex_paginatitulo_mod_OV01.tex (fin)

% Para portada: latex_paginatitulo_mod_OV03.tex (inicio)
\usepackage{KTHEEtitlepage}
% Para portada: latex_paginatitulo_mod_OV03.tex (fin)

\renewcommand{\contentsname}{Índice}
\renewcommand{\listfigurename}{Índice de figuras}
\renewcommand{\listtablename}{Índice de tablas}
\newcommand{\bcols}{}
\newcommand{\ecols}{}
\newcommand{\bcol}[1]{\begin{minipage}{#1\linewidth}}
\newcommand{\ecol}{\end{minipage}}
\newcommand{\balertblock}[1]{\begin{alertblock}{#1}}
\newcommand{\ealertblock}{\end{alertblock}}
\newcommand{\bitemize}{\begin{itemize}}
\newcommand{\eitemize}{\end{itemize}}
\newcommand{\benumerate}{\begin{enumerate}}
\newcommand{\eenumerate}{\end{enumerate}}
\newcommand{\saltopagina}{\newpage}
\newcommand{\bcenter}{\begin{center}}
\newcommand{\ecenter}{\end{center}}
\newcommand{\beproof}{\begin{proof}} %new
\newcommand{\eeproof}{\end{proof}} %new
%De: https://texblog.org/2007/11/07/headerfooter-in-latex-with-fancyhdr/
% \fancyhead
% E: Even page
% O: Odd page
% L: Left field
% C: Center field
% R: Right field
% H: Header
% F: Footer
%\fancyhead[CO,CE]{Resultados}

%OPCION 1
% \fancyhead[LE,RO]{\slshape \rightmark}
% \fancyhead[LO,RE]{\slshape \leftmark}
% \fancyfoot[C]{\thepage}
% \renewcommand{\headrulewidth}{0.4pt}
% \renewcommand{\footrulewidth}{0pt}

%OPCION 2
% \fancyhead[LE,RO]{\slshape \rightmark}
% \fancyfoot[LO,RE]{\slshape \leftmark}
% \fancyfoot[LE,RO]{\thepage}
% \renewcommand{\headrulewidth}{0.4pt}
% \renewcommand{\footrulewidth}{0.4pt}
%%%%%%%%%%
\usepackage{calc,amsfonts}
% Elimina la cabecera de páginas impares vacías al finalizar los capítulos
\usepackage{emptypage}
\makeatletter

%\definecolor{ocre}{RGB}{25,25,243} % Define el color azul (naranja) usado para resaltar algunas salidas
\definecolor{ocre}{RGB}{0,0,0} % Define el color a negro (aparece en los teoremas

%\usepackage{calc} 


%era if(csl-refs) con dolares
% metodobib: true


\usepackage{lipsum}

%\usepackage{tikz} % Requerido para dibujar formas personalizadas

%\usepackage{amsmath,amsthm,amssymb,amsfonts}
\usepackage{amsthm}


% Boxed/framed environments
\newtheoremstyle{ocrenumbox}% % Theorem style name
{0pt}% Space above
{0pt}% Space below
{\normalfont}% % Body font
{}% Indent amount
{\small\bf\sffamily\color{ocre}}% % Theorem head font
{\;}% Punctuation after theorem head
{0.25em}% Space after theorem head
{\small\sffamily\color{ocre}\thmname{#1}\nobreakspace\thmnumber{\@ifnotempty{#1}{}\@upn{#2}}% Theorem text (e.g. Theorem 2.1)
\thmnote{\nobreakspace\the\thm@notefont\sffamily\bfseries\color{black}---\nobreakspace#3.}} % Optional theorem note
\renewcommand{\qedsymbol}{$\blacksquare$}% Optional qed square

\newtheoremstyle{blacknumex}% Theorem style name
{5pt}% Space above
{5pt}% Space below
{\normalfont}% Body font
{} % Indent amount
{\small\bf\sffamily}% Theorem head font
{\;}% Punctuation after theorem head
{0.25em}% Space after theorem head
{\small\sffamily{\tiny\ensuremath{\blacksquare}}\nobreakspace\thmname{#1}\nobreakspace\thmnumber{\@ifnotempty{#1}{}\@upn{#2}}% Theorem text (e.g. Theorem 2.1)
\thmnote{\nobreakspace\the\thm@notefont\sffamily\bfseries---\nobreakspace#3.}}% Optional theorem note

\newtheoremstyle{blacknumbox} % Theorem style name
{0pt}% Space above
{0pt}% Space below
{\normalfont}% Body font
{}% Indent amount
{\small\bf\sffamily}% Theorem head font
{\;}% Punctuation after theorem head
{0.25em}% Space after theorem head
{\small\sffamily\thmname{#1}\nobreakspace\thmnumber{\@ifnotempty{#1}{}\@upn{#2}}% Theorem text (e.g. Theorem 2.1)
\thmnote{\nobreakspace\the\thm@notefont\sffamily\bfseries---\nobreakspace#3.}}% Optional theorem note

% Non-boxed/non-framed environments
\newtheoremstyle{ocrenum}% % Theorem style name
{5pt}% Space above
{5pt}% Space below
{\normalfont}% % Body font
{}% Indent amount
{\small\bf\sffamily\color{ocre}}% % Theorem head font
{\;}% Punctuation after theorem head
{0.25em}% Space after theorem head
{\small\sffamily\color{ocre}\thmname{#1}\nobreakspace\thmnumber{\@ifnotempty{#1}{}\@upn{#2}}% Theorem text (e.g. Theorem 2.1)
\thmnote{\nobreakspace\the\thm@notefont\sffamily\bfseries\color{black}---\nobreakspace#3.}} % Optional theorem note
\renewcommand{\qedsymbol}{$\blacksquare$}% Optional qed square
\makeatother



% Define el estilo texto theorem para cada tipo definido anteriormente
\newcounter{dummy} 
\numberwithin{dummy}{section}
\theoremstyle{ocrenumbox}
\newtheorem{theoremeT}[dummy]{Teorema}  % (Pedro: Theorem)
\newtheorem{problem}{Problema}[chapter]  % (Pedro: Problem)
\newtheorem{exerciseT}{Ejercicio}[chapter] % (Pedro: Exercise)
\theoremstyle{blacknumex}
\newtheorem{exampleT}{Ejemplo}[chapter] % (Pedro: Example)
\theoremstyle{blacknumbox}
\newtheorem{vocabulary}{Vocabulario}[chapter]  % (Pedro: Vocabulary)
\newtheorem{definitionT}{Definición}[section]  % (Pedro: Definition)
\newtheorem{corollaryT}[dummy]{Corolario}  % (Pedro: Corollary)
\theoremstyle{ocrenum}
\newtheorem{proposition}[dummy]{Proposición} % (Pedro: Proposition)


\usepackage[framemethod=default]{mdframed}



\newcommand{\intoo}[2]{\mathopen{]}#1\,;#2\mathclose{[}}
\newcommand{\ud}{\mathop{\mathrm{{}d}}\mathopen{}}
\newcommand{\intff}[2]{\mathopen{[}#1\,;#2\mathclose{]}}
\newtheorem{notation}{Notation}[chapter]


\mdfdefinestyle{exampledefault}{%
rightline=true,innerleftmargin=10,innerrightmargin=10,
frametitlerule=true,frametitlerulecolor=green,
frametitlebackgroundcolor=yellow,
frametitlerulewidth=2pt}


% Theorem box
\newmdenv[skipabove=7pt,
skipbelow=7pt,
backgroundcolor=black!5,
linecolor=ocre,
innerleftmargin=5pt,
innerrightmargin=5pt,
innertopmargin=10pt,%5pt
leftmargin=0cm,
rightmargin=0cm,
innerbottommargin=5pt]{tBox}

% Exercise box	  
\newmdenv[skipabove=7pt,
skipbelow=7pt,
rightline=false,
leftline=true,
topline=false,
bottomline=false,
backgroundcolor=ocre!10,
linecolor=ocre,
innerleftmargin=5pt,
innerrightmargin=5pt,
innertopmargin=10pt,%5pt
innerbottommargin=5pt,
leftmargin=0cm,
rightmargin=0cm,
linewidth=4pt]{eBox}	

% Definition box
\newmdenv[skipabove=7pt,
skipbelow=7pt,
rightline=false,
leftline=true,
topline=false,
bottomline=false,
linecolor=ocre,
innerleftmargin=5pt,
innerrightmargin=5pt,
innertopmargin=10pt,%0pt
leftmargin=0cm,
rightmargin=0cm,
linewidth=4pt,
innerbottommargin=0pt]{dBox}	

% Corollary box
\newmdenv[skipabove=7pt,
skipbelow=7pt,
rightline=false,
leftline=true,
topline=false,
bottomline=false,
linecolor=gray,
backgroundcolor=black!5,
innerleftmargin=5pt,
innerrightmargin=5pt,
innertopmargin=10pt,%5pt
leftmargin=0cm,
rightmargin=0cm,
linewidth=4pt,
innerbottommargin=5pt]{cBox}

% Crea un entorno para cada tipo de theorem y le asigna un estilo 
% con ayuda de las cajas coloreadas anteriores
\newenvironment{theorem}{\begin{tBox}\begin{theoremeT}}{\end{theoremeT}\end{tBox}}
\newenvironment{exercise}{\begin{eBox}\begin{exerciseT}}{\hfill{\color{ocre}\tiny\ensuremath{\blacksquare}}\end{exerciseT}\end{eBox}}				  
\newenvironment{definition}{\begin{dBox}\begin{definitionT}}{\end{definitionT}\end{dBox}}	
\newenvironment{example}{\begin{exampleT}}{\hfill{\tiny\ensuremath{\blacksquare}}\end{exampleT}}		
\newenvironment{corollary}{\begin{cBox}\begin{corollaryT}}{\end{corollaryT}\end{cBox}}	

%	ENVIRONMENT remark
\newenvironment{remark}{\par\vspace{10pt}\small 
% Espacio blanco vertical sobre la nota y tamaño de fuente menor
\begin{list}{}{
\leftmargin=35pt % Indentación sobre la izquierda
\rightmargin=25pt}\item\ignorespaces % Indentación sobre la derecha
\makebox[-2.5pt]{\begin{tikzpicture}[overlay]
\node[draw=ocre!60,line width=1pt,circle,fill=ocre!25,font=\sffamily\bfseries,inner sep=2pt,outer sep=0pt] at (-15pt,0pt){\textcolor{ocre}{N}}; \end{tikzpicture}} % R naranja en un círculo (Pedro)
\advance\baselineskip -1pt}{\end{list}\vskip5pt} 
% Espaciado de línea más estrecho y espacio en blanco después del comentario


\newenvironment{solutionExe}{\par\vspace{10pt}\small 
\begin{list}{}{
\leftmargin=35pt 
\rightmargin=25pt}\item\ignorespaces 
\makebox[-2.5pt]{\begin{tikzpicture}[overlay]
\node[draw=ocre!60,line width=1pt,circle,fill=ocre!25,font=\sffamily\bfseries,inner sep=2pt,outer sep=0pt] at (-15pt,0pt){\textcolor{ocre}{S}}; \end{tikzpicture}} 
\advance\baselineskip -1pt}{\end{list}\vskip5pt} 

\newenvironment{solutionExa}{\par\vspace{10pt}\small 
\begin{list}{}{
\leftmargin=35pt 
\rightmargin=25pt}\item\ignorespaces 
\makebox[-2.5pt]{\begin{tikzpicture}[overlay]
\node[draw=ocre!60,line width=1pt,circle,fill=ocre!55,font=\sffamily\bfseries,inner sep=2pt,outer sep=0pt] at (-15pt,0pt){\textcolor{ocre}{S}}; \end{tikzpicture}} 
\advance\baselineskip -1pt}{\end{list}\vskip5pt} 

\usepackage{tcolorbox}

\usetikzlibrary{trees}

\theoremstyle{ocrenum}
\newtheorem{solutionT}[dummy]{Solución}  % (Pedro: Corollary)
\newenvironment{solution}{\begin{cBox}\begin{solutionT}}{\end{solutionT}\end{cBox}}	


\newcommand{\tcolorboxsolucion}[2]{%
\begin{tcolorbox}[colback=green!5!white,colframe=green!75!black,title=#1] 
 #2
 %\tcblower  % pone una línea discontinua
\end{tcolorbox}
}% final definición comando

\newtcbox{\mybox}[1][green]{on line,
arc=0pt,outer arc=0pt,colback=#1!10!white,colframe=#1!50!black, boxsep=0pt,left=1pt,right=1pt,top=2pt,bottom=2pt, boxrule=0pt,bottomrule=1pt,toprule=1pt}



\mdfdefinestyle{exampledefault}{%
rightline=true,innerleftmargin=10,innerrightmargin=10,
frametitlerule=true,frametitlerulecolor=green,
frametitlebackgroundcolor=yellow,
frametitlerulewidth=2pt}





\newcommand{\betheorem}{\begin{theorem}}
\newcommand{\eetheorem}{\end{theorem}}
\newcommand{\bedefinition}{\begin{definition}}
\newcommand{\eedefinition}{\end{definition}}

\newcommand{\beremark}{\begin{remark}}
\newcommand{\eeremark}{\end{remark}}
\newcommand{\beexercise}{\begin{exercise}}
\newcommand{\eeexercise}{\end{exercise}}
\newcommand{\beexample}{\begin{example}}
\newcommand{\eeexample}{\end{example}}
\newcommand{\becorollary}{\begin{corollary}}
\newcommand{\eecorollary}{\end{corollary}}


\newcommand{\besolutionExe}{\begin{solutionExe}}
\newcommand{\eesolutionExe}{\end{solutionExe}}
\newcommand{\besolutionExa}{\begin{solutionExa}}
\newcommand{\eesolutionExa}{\end{solutionExa}}


%%%%%%%%


% Caja Salida Markdown
\newmdenv[skipabove=7pt,
skipbelow=7pt,
rightline=false,
leftline=true,
topline=false,
bottomline=false,
backgroundcolor=GreenYellow!10,
linecolor=GreenYellow!80,
innerleftmargin=5pt,
innerrightmargin=5pt,
innertopmargin=10pt,%5pt
innerbottommargin=5pt,
leftmargin=0cm,
rightmargin=0cm,
linewidth=4pt]{mBox}	

%% RMarkdown
\newenvironment{markdownsal}{\begin{mBox}}{\end{mBox}}	

\newcommand{\bmarkdownsal}{\begin{markdownsal}}
\newcommand{\emarkdownsal}{\end{markdownsal}}


\usepackage{array}
\usepackage{multirow}
\usepackage{wrapfig}
\usepackage{colortbl}
\usepackage{pdflscape}
\usepackage{tabu}
\usepackage{threeparttable}
\usepackage{subfig} %new
%\usepackage{booktabs,dcolumn,rotating,thumbpdf,longtable}
\usepackage{dcolumn,rotating}  %new
\usepackage[graphicx]{realboxes} %new de: https://stackoverflow.com/questions/51633434/prevent-pagebreak-in-kableextra-landscape-table

%define el interlineado vertical
%\renewcommand{\baselinestretch}{1.5}

%define etiqueta para las Tablas o Cuadros
%\renewcommand\spanishtablename{Tabla}

%%\bibliographystyle{plain} %new no necesario


%%%%%%%%%%%% PARA USO CON biblatex
% \DefineBibliographyStrings{english}{%
%   backrefpage = {ver pag.\adddot},%
%   backrefpages = {ver pags.\adddot}%
% }

% \DefineBibliographyStrings{spanish}{%
%   backrefpage = {ver pag.\adddot},%
%   backrefpages = {ver pags.\adddot}%
% }
% 
% \DeclareFieldFormat{pagerefformat}{\mkbibparens{{\color{red}\mkbibemph{#1}}}}
% \renewbibmacro*{pageref}{%
%   \iflistundef{pageref}
%     {}
%     {\printtext[pagerefformat]{%
%        \ifnumgreater{\value{pageref}}{1}
%          {\bibstring{backrefpages}\ppspace}
%          {\bibstring{backrefpage}\ppspace}%
%        \printlist[pageref][-\value{listtotal}]{pageref}}}}
% 
%%% de kableExtra
\usepackage{booktabs}
\usepackage{longtable}
%\usepackage{array}
%\usepackage{multirow}
%\usepackage{wrapfig}
%\usepackage{float}
%\usepackage{colortbl}
%\usepackage{pdflscape}
%\usepackage{tabu}
%\usepackage{threeparttable}
\usepackage{threeparttablex}
\usepackage[normalem]{ulem}
\usepackage{makecell}
%\usepackage{xcolor}

%%%%%%% fin: latex_preambulo.tex PLLC


\begin{document}

\bibliographystyle{flexbib}



\raggedbottom

\ifdefined\ifprincipal
\else
\setlength{\parindent}{1em}
\pagestyle{fancy}
\setcounter{tocdepth}{4}
\tableofcontents

\fi

\ifdefined\ifdoblecara
\fancyhead{}{}
\fancyhead[LE,RO]{\scriptsize\rightmark}
\fancyfoot[LO,RE]{\scriptsize\slshape \leftmark}
\fancyfoot[C]{}
\fancyfoot[LE,RO]{\footnotesize\thepage}
\else
\fancyhead{}{}
\fancyhead[RO]{\scriptsize\rightmark}
\fancyfoot[LO]{\scriptsize\slshape \leftmark}
\fancyfoot[C]{}
\fancyfoot[RO]{\footnotesize\thepage}
\fi

\renewcommand{\headrulewidth}{0.4pt}
\renewcommand{\footrulewidth}{0.4pt}

\hypertarget{conclusiones-aportaciones-y-trabajo-futuro}{%
\chapter{Conclusiones, aportaciones y trabajo
futuro}\label{conclusiones-aportaciones-y-trabajo-futuro}}

En este último capítulo se va a realizar una recapitulación de las
conclusiones extraídas en cada una de las secciones del presente trabajo
de fin de estudios. En primer lugar, se presentarán de forma resumida
las conclusiones obtenidas a lo largo del trabajo. A continuación, se
detallarán las aportaciones realizadas en el campo de la predicción de
incendios forestales. Por último, se indicarán algunas lineas de
investigación, dentro del campo de la predicción de incendios forestales
y la inteligencia artificial, que permitirían profundizar en el
desarrollo de la metodología presentada de cara a obtener mejores
modelos con utilidad práctica.

\hypertarget{conclusiones}{%
\section{Conclusiones}\label{conclusiones}}

El control y extinción de los incendios forestales requiere del
desplazamiento de un gran número de efectivos, y el éxito en la
operación depende en muchos casos de la velocidad en la respuesta. En el
presente trabajo se ha desarrollado una metodología completa para dotar
de herramientas que permitan predecir eficazmente las zonas en riesgo de
verse afectadas por un incendio forestal en la Comunidad Autónoma de
Andalucía, facilitando así la toma de decisiones y permitiendo una
asignación de los recursos más eficiente.

A lo largo de esta memoria se ha desarrollado una metodología completa
para la construcción de modelos de predicción de incendios forestales en
la Comunidad Autónoma de Andalucía mediante el uso de técnicas de
\emph{Machine Learning} y procesamiento de datos geoespaciales,
adoptando un enfoque dinámico y global. Dinámico, pues se predice el el
riesgo de incendio forestal para una localización específica en un día
concreto. Global, pues se han considerado 27 variables que abarcan las 5
dimensiones principales : antropológica, demográfica, meteorológica,
topográfica y de vegetación.

Se ha comenzado introduciendo el problema, justificando su relevancia y
estableciendo 3 tareas claras para alcanzar el objetivo del trabajo. La
primera de las tareas implicaba la construcción de un conjunto de datos
adecuado para el análisis estadístico y la construcción de modelos de
ML. Esta tarea se abordó en el capítulo 2, donde no solo se ha generado
un conjunto de 20.000 muestras sobre el que se ha desarrollado el
trabajo, si no que se ha implementado un algoritmo para tomar muestras
aleatorias de casos positivos y negativos dentro del marco del estudio y
asociar a cada observación los valores correspondientes de todas las
variables consideradas. Se ha explorado el uso de estratificación por
mes en la selección de la muestra, con el objetivo de entrenar modelos
que sean capaces de detectar relaciones más profundas en los datos
relacionadas con la aparición de los incendios forestales. Se ha visto
que, por ejemplo, en el caso del \emph{Random Forest} esta
estratificación ha conducido a que la variable \emph{uso\_suelo} haya
tomado demasiada importancia en la clasificación, conduciendo a modelos
que no parecen comportarse adecuadamente. \textbar\textbar\textbar{}
Queda, por tanto, abierta la pregunta de cuál sería la mejor manera de
tomar la muestra de casos negativos de forma que se encuentre un
equilibrio entre la sensibilidad del modelo a las variables
meteorológicas sin perder la influencia de las demás variables.
\textbar\textbar\textbar{}

A continuación, se ha analizado en profundidad el conjunto de datos
generado, recurriendo principalmente a representaciones gráficas, aunque
también se han usado métodos numéricos. La complejidad de esta fase
radica en que para llegar a obtener información relevante es necesario
tener en cuenta las dimensiones espacial y temporal de los datos. Este
proceso ha permitido llegar a un mayor conocimiento acerca del conjunto
de datos, caracterizado por una gran presencia de valores
\emph{outlier}, por distribuciones asimétricas hacia la derecha en casi
todas las variables numéricas y correlaciones generalmente pequeñas en
valor absoluto. A nivel gráfico, se ha observado que las variables que
muestran mayores diferencias entre ambas clases son \emph{PRECTOTCORR}
(el total diario de precipitaciones), \emph{WS10M} (la velocidad del
viento a 10 m) y \emph{uso\_suelo} (la clasificación de uso de suelo).
Un buen reflejo de la complejidad del conjunto de datos es que para
explicar el \(80%
\) de la varianza de las 18 variables numéricas en la muestra se
necesitan 11 componentes principales.

La siguiente tarea planteada se ha desarrollado en el capítulo 5, donde
se han construido distintos modelos de ML de clasificación binaria. Se
ha usado el flujo de trabajo propuesto por el paquete de R
\emph{tidymodels} para preprocesar los datos, entrenar los modelos,
ajustar los valores de los hiperparámetros y evaluar sus rendimientos en
los datos test. Los modelos considerados han sido: regresión logística
con penalización, regresión logística con penalización usando PCA,
k-\emph{Nearest Neighbours}, SVM lineal, SVM radial, árbol de decisión y
\emph{Random Forest}. Se ha usado una partición temporal en
entrenamiento-validación-test para entrenar los modelos, ajustar los
parámetros y evaluar su capacidad de generalización sobre nuevos datos.
Los mejores resultados sobre el conjunto de datos generado con
estratificación por mes, los han dado los modelos de regresión logística
con penalización, SVM lineal y SVM radial, los cuales han mostrado un
comportamiento muy similar tanto sobre el conjunto de validación como
sobre el conjunto test. Los valores más elevados en las métricas de de
rendimiento sobre los datos test los ha alcanzado el modelo de regresión
logística lasso con un coeficiente de penalización de 0.000464, que ha
alcanzado un ROC-AUC de 0.795, una tasa de acierto de 0.710, 0.726, una
especificidad de 0.693 y una presición de 0.718.

Finalmente, se ha evaluado el desempeño de los modelos en dos casos
prácticos, cumpliendo así con la tercera tarea planteada. Primero, se
han usado los modelos de regresión logistica con penalización, SVM
lineal y RF para predecir la probabilidad estimada de incendio el día 15
de cada mes de 2022 en toda Andalucía. Y por último, se han usado el
modelo de regresión logística lasso y el SVM lineal para para predecir
el incendio provocado de Sierra Bermeja, en Málaga. Aunque la baja
resolución de las variables meteorológicas impide llegar a un mayor
nivel de detalle, ambos modelos están indicando una situación de riesgo
elevado de incendio forestal en la zona el día 8 de septiembre de 2023,
al observarse un incremento significativo de las probabilidades de
incendio estimadas en todo el área en estudio, como consecuencia de unas
condiciones meteorológicas especialmente favorables para un incendio
forestal.

Es necesario mencionar también las limitaciones de los modelos
construidos, aunque algunas de ellas ya han sido comentadas a lo largo
del trabajo. Por un lado, hay que tener ciertas consideraciones sobre
los datos considerados para el trabajo. Aunque las variables
consideradas son fruto de un amplio estudio previo y cobren las 5
dimensiones consideradas, son solo una primera aproximación aproximación
al problema y probablemente en futuros estudios se necesite considerar
un número mucho más elevado de variables, dada la complicada naturaleza
del problema. Las información de los incendios de la que se ha podido
disponer solo cubre los incendios que calcinaron una extensión superior
a 100 ha, no siendo así exhaustiva, por lo que las conclusiones que
puedan extraerse de los modelos construidos son limitadas. Además, como
ya se ha indicado, la calidad de la información meteorológica disponible
es limitada al proceder de modelos construidos a partir de observaciones
satelitales.

Por otro lado, desde un punto de vista estadístico, es importante
mencionar que la hipótesis de independencia entre las observaciones,
sobre la que se sustentan los modelos construidos es violada fuertemente
por los datos, al estar correlacionados espacial y temporalmente. Sin
embargo, esto no impide su utilización, ya que de hecho han alcanzado
resultados bastante satisfactorios. Otro comentario en esta línea es que
todos los modelos construidos realizan la predicción de forma local, sin
tener en cuenta la estructura de correlaciones presente en los datos, lo
que inevitablemente conlleva un pérdida de información importante. Por
ello, explorar el uso de modelos más complejos que sí consideren las
correlaciones temporales y espaciales existentes entre las observaciones
podría llevar a alcanzar mejores resultados.

Cabe también mencionar que las estimaciones de error con las que se han
evaluado los modelos son medidas globales obtenidas en una muestra
concreta, por lo que deben ser tomadas con precaución, tanto en este
estudio, como en todos los que se aborden problemas similares. Por ello
se han realizado análisis prácticos del desempeño de los modelos, para
evaluar su rendimiento en la realidad. Por último, se debe considerar
también que al ser un problema tan complejo en el que influyen tantas
dimensiones diferentes, la significación que pueda tener un solo valor
es limitada, por muy bueno que sea el modelo. Así, las conclusiones de
los modelos deberán ser siempre tomadas con precaución y bajo el
asesoramiento de un experto en el incendios forestales.

Pese a todo ello, los resultados obtenidos en los modelos construidos a
lo largo de este trabajo son prometedores, ilustrando el potencial que
podría llegar a tener la aplicación de la Inteligencia Artificial en la
predicción de incendios forestales. Sin embargo, se trata de una
investigación introductoria limitada por los recursos disponibles.
Aumentar el número de variables consideradas, obtener fuentes de
información de mayor calidad, considerar modelos más complejos, estudiar
la forma óptima de generar la muestra de casos negativos para entrenar
los modelos son algunas de las tareas que deberán llevarse a cabo en
futuras investigaciones para permitir que estas tecnologías tengan un
impacto en la lucha contra el fuego, permitiendo una mejor gestión de
los recursos y llegando a salvar vidas.

\hypertarget{aportaciones}{%
\section{Aportaciones}\label{aportaciones}}

A lo largo de la presente memoria se ha presentado una metodología
completa para construir modelos de predicción de incendios forestales
desde un enfoque dinámico y global. En este trabajo se ha aplicado al
caso de la Comunidad Autónoma de Andalucía considerando un conjunto de
27 variables explicativas, aunque su extensión a otras regiones y a
conjuntos de variables más amplios es relativamente sencillo. Las
principales aportaciones del proyecto han sido:

\begin{itemize}
\item
  La recopilación de una gran cantidad de información relevante para el
  estudio de los incendios forestales a partir de fuentes oficiales
  (Tabla \ref{tab:fuentes}).
\item
  La implementación de métodos y funciones para procesar los conjuntos
  de datos espaciales recopilados y generar muestras útiles para el
  análisis estadístico y la construcción de modelos de clasificación
  binaria, asociando a cada observación los valores correspondientes de
  todas las variables explicativas consideras. Estas funciones pueden
  ser útiles también fuera del ámbito de estudio de los incendios
  forestales, ya que permiten conocer los valores de las 27 variables
  consideradas correspondientes a un día y una localización dentro de
  los límites del estudio.
\item
  El uso de estratificación en el proceso de generación de la muestra,
  con el objetivo de evitar modelos superficiales y estimaciones del
  error sesgadas positivamente. Aunque los resultados obtenidos, no son
  del todo satisfactorios y requieren de un mayor estudio, de entre
  todos los trabajos similares consultados, este es el primero en el que
  se cuestiona la forma de tomar las muestras negativas.
\item
  La generación de una gran cantidad de mapas y gráficos que permiten
  estudiar en profundidad la distribución espacio-temporal de las
  variables incluidas en el estudio.
\item
  El entrenamiento, ajuste y comparación del rendimiento de distintos
  algoritmos de clasificación binaria dentro del ML.
\end{itemize}

\hypertarget{trabajo-futuro}{%
\section{Trabajo futuro}\label{trabajo-futuro}}

En el presente trabajo se han llegado a resultados prometedores en
cuanto al potencial de las herramientas en él presentadas. Sin embargo,
es incuestionable que las limitaciones en tiempo y recursos han obligado
a adoptar un enfoque más global, tomando ciertas simplificaciones y
dejando algunos caminos sin explorar o sin estudiar en toda la
profundidad que requieren. Es por ello que este estudio no está
completo. Para poder llegar a construir modelos que verdaderamente sean
útiles en el campo de la predicción de incendios forestales es necesario
ahondar en esta investigación y dedicarle una mayor cantidad de
recursos.

A lo largo del trabajo se han justificado todas las decisiones
metodológicas tomadas, indicando en muchos casos alternativas mejores a
las empleadas pero inviables dadas las limitaciones de un trabajo de fin
de estudios. A continuación se recopilan estas propuestas, añadiendo
algunas otras, con el objetivo de mostrar lineas posibles de
investigación para extender la metodología presentada, llegando así a
construir modelos que puedan tener un impacto significativo en la lucha
contra el fuego:

\begin{itemize}
\item
  Aumentar el número de variables consideradas, tomando como referencia
  trabajos como \citet{HumanCauseWildFiresSpain} y \citet{logreg_hcwf},
  en los que se estudian las variables más influyentes en la predicción
  de incendios causados por el hombre.
\item
  Explorar el uso de la EGIF para la obtención de la información
  relativa a los incendios forestales. Esto es de vital importancia de
  cara a extender el estudio, ya que en el estudio presente solo se han
  considerado los incendios mayores de 100 ha, puesto que son los únicos
  disponibles en la REDIAM. Además, esto permitiría añadir otras
  dimensiones al problema, como la predicción de la superficie afectada
  por los incendios forestales o de la propagación de los incendios a
  partir de los puntos de origen del fuego.
\item
  Buscar fuentes de información meteorológica viables y de mayor
  calidad, a ser posible proveniente de estaciones meteorológicas y no
  de modelos basado en observaciones . En esta dirección podría ser
  interesante explorar en mayor profundidad el uso de la API de la
  AEMET.
\item
  Sería necesario revisar el procedimiento de generación de la muestra
  de de casos negativos, ajustando convenientemente los parámetros
  considerados. Como ha quedado reflejado en el trabajo, la composición
  de la muestra de casos negativos usada para entrenar los modelos tiene
  un impacto directo en el funcionamiento de estos. De la misma forma
  aquí se ha considerado un muestreo con estratificación por mes y un
  muestreo completamente aleatorio en las fechas, sería interesante
  estudiar los efectos que pueden tener otras formas de seleccionar la
  muestra de casos negativos. Esto será especialmente relevante si los
  tamaños muestrales son reducidos.
\item
  El uso de modelos de \emph{Deep Learning} como redes convolucionales
  podría traer mejoras significativas en los modelos, al considerar la
  estructura de correlaciones espaciales presentes en los datos
  \citep{SpainOnFire}.
\end{itemize}

\bibliography{bib/library.bib,bib/paquetes.bib}


%


\end{document}
